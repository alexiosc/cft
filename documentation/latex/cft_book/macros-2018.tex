% -*- latex -*-


%% %%%%%%%%%%%%%%%%%%%%%%%%%%%%%%%%%%%%%%%%%%%%%%%%%%%%%%%%%%%%%%%%%%%%%%%%%%%%%%%
%% %%
%% %% WORKAROUNDS
%% %%
%% %%%%%%%%%%%%%%%%%%%%%%%%%%%%%%%%%%%%%%%%%%%%%%%%%%%%%%%%%%%%%%%%%%%%%%%%%%%%%%%

%% %% \ifxetex
%% %%   % XeTeX doesn't have HTML output, of course.
%% %%   \def\HCode#1{}
%% %% \else
%% %%   % This is a hack.
%% %%   \newcounter{Hfootnote}
%% %% \fi



%% %%%%%%%%%%%%%%%%%%%%%%%%%%%%%%%%%%%%%%%%%%%%%%%%%%%%%%%%%%%%%%%%%%%%%%%%%%%%%%%
%% %%
%% %% HTML GENERATION
%% %%
%% %%%%%%%%%%%%%%%%%%%%%%%%%%%%%%%%%%%%%%%%%%%%%%%%%%%%%%%%%%%%%%%%%%%%%%%%%%%%%%%

%% \ifxetex
%%   \def\texonly#1{#1}
%%   \def\htmlonly#1{}
%%   \newenvironment{htmldiv}[1]{}{}
%%   \newcommand{\htmlspan}[2]{#2}
%%   \newcommand{\htmlbreak}{}
%% \else
%%   \def\texonly#1{}
%%   \def\htmlonly#1{#1}
%%   \newenvironment{htmldiv}[1]{\HCode{<div class="#1">}}{\HCode{</div>}}
%%   \newcommand{\htmlspan}[2]{\HCode{<span class="#1">}{#2}\HCode{</span>}}
%%   \newcommand{\htmlbreak}{\HCode{<div class="break" />}}
%% \fi


%% %%%%%%%%%%%%%%%%%%%%%%%%%%%%%%%%%%%%%%%%%%%%%%%%%%%%%%%%%%%%%%%%%%%%%%%%%%%%%%%
%% %%
%% %% LINKING TO LOCATIONS IN THE DOCUMENT
%% %%
%% %%%%%%%%%%%%%%%%%%%%%%%%%%%%%%%%%%%%%%%%%%%%%%%%%%%%%%%%%%%%%%%%%%%%%%%%%%%%%%%

%% % Use hyperlinking when rendering PDFs
%% \newcommand{\barecf}[1]{\hyperref[#1]{\ref*{#1}}}

%% \newcommand{\cf}[2][section]{\hyperref[#2]{%
%%   \ifxetex%
%%     \ifthenelse{\equal{\pageref*{#2}}{\thepage}}%
%%                {#1 \ref*{#2}}%
%%                {#1 \ref*{#2} (p.~\pageref*{#2})}%
%%   \else%
%%                {#1 \ref*{#2}}%
%%   \fi%
%% }}

%% \newcommand{\cfp}[2][section]{\hyperref[#2]{%
%%   \ifxetex%
%%     \ifthenelse{\equal{\pageref*{#2}}{\thepage}}%
%%       {#1 \ref*{#2}}%
%%       {#1 \ref*{#2}, p.~\pageref*{#2}}%
%%   \else
%%      {#1 \ref*{#2}}%
%%   \fi%
%% }}

%% \newcommand{\fcf}[1]{\cf[figure]{#1}}
%% \newcommand{\fcfp}[1]{\cfp[figure]{#1}}
%% \newcommand{\tcf}[1]{\cf[table]{#1}}
%% \newcommand{\tcfp}[1]{\cfp[table]{#1}}
%% \newcommand{\ccf}[1]{\cf[chapter]{#1}}
%% \newcommand{\ccfp}[1]{\cfp[chapter]{#1}}
%% %
%% \newcommand{\npcf}[2][section]{\hyperref[#2]{#1 \ref*{#2}}}
%% \newcommand{\appcf}[1]{\cf[appendix]{#1}}
%% \newcommand{\ecf}[1]{\cf[equation]{#1}}
%% \newcommand{\algcf}[1]{\cf[algorithm]{#1}}
%% \newcommand{\npappcf}[1]{\npcf[appendix]{#1}}
%% \newcommand{\npccf}[1]{\npcf[chapter]{#1}}
%% \newcommand{\npfcf}[1]{\npcf[figure]{#1}}
%% \newcommand{\nptcf}[1]{\npcf[table]{#1}}
%% \newcommand{\npecf}[1]{\npcf[equation]{#1}}
%% \newcommand{\npalgcf}[1]{\npcf[algorithm]{#1}}



%% %%%%%%%%%%%%%%%%%%%%%%%%%%%%%%%%%%%%%%%%%%%%%%%%%%%%%%%%%%%%%%%%%%%%%%%%%%%%%%%
%% %%
%% %% LISTS OF THINGS
%% %%
%% %%%%%%%%%%%%%%%%%%%%%%%%%%%%%%%%%%%%%%%%%%%%%%%%%%%%%%%%%%%%%%%%%%%%%%%%%%%%%%%

%% %% %%%\addtolength\cftfignumwidth{1.5em}
%% \ifxetex
%%   \makeatletter
%%   \renewcommand*\l@section{\@dottedtocline{1}{1.5em}{2.3em}}
%%   \renewcommand*\l@subsection{\@dottedtocline{2}{3.8em}{3.2em}}
%%   \renewcommand*\l@subsubsection{\@dottedtocline{3}{7.0em}{4.1em}}
%%   \renewcommand*\l@paragraph{\@dottedtocline{4}{10em}{5em}}
%%   \renewcommand*\l@subparagraph{\@dottedtocline{5}{12em}{6em}}
%%   \setcounter{maxsecnumdepth}{3}

%%   \renewcommand{\@pnumwidth}{3em}
%%   \renewcommand{\@tocrmarg}{4em}
%%   \makeatother
%% \else
%%   \relax
%% \fi

%% %
%% % Schematics
%% %

%% \newcommand\listschematicname{List of Schematics} 
%% \newcommand{\schematic}[1]{%
%%   \refstepcounter{schematic}%
%%   \par\noindent\textbf{Schematic \theschematic. #1}
%%   \addcontentsline{los}{section}{\protect\numberline{\theschematic}#1}\par%
%% }
%% \newcommand{\listofschematics}{\listofschematic}

%% %
%% % I/O Ports
%% %

%% \newcommand\listioportname{List of Input/Output Ports} 
%% \newlistof{listofioport}{loioport}{\listioportname}
%% %% \newcommand{\registerioport}[1]{%
%% %%   \refstepcounter{ioport}%
%% %%   \addcontentsline{loioport}{section}{\protect\numberline{\ }%
%% %% }
%% \newcommand\listofioports\listofioport



%% \newcommand\caution[1]{\textcolor{caution}{\textbf{#1}}}
%% %\newcommand\todo[1]{\textcolor{caution}{\bf{TODO: #1}}}

%% \ifxetex
%%   \newenvironment{obsoleted}{}{}
%% \else
%%   \newenvironment{obsoleted}{\begin{htmldiv}{obsoleted box}}{\end{htmldiv}}
%% \fi

%% %
%% % Tasks
%% %

%% \newcommand\listtasksname{List of Incomplete Tasks} 
%% \newlistof{listoftask}{lotasks}{\listtasksname}
%% \newcounter{task}

%% \ifxetex
%%   \newcommand{\todo}[1]{%
%%     \refstepcounter{task}%
%%     {\textcolor{caution}{\textbf{TODO: #1}}}
%%     \addcontentsline{lotasks}{section}{\protect\numberline{\arabic{task}}To Do: #1}\par%
%%   }
%%   \newcommand{\bug}[2]{%
%%     \refstepcounter{task}%
%%     {\textcolor{caution}{\textbf{BUG: #1 #2}}}
%%     \addcontentsline{lotasks}{section}{\protect\numberline{\arabic{task}}Bug: #1}\par%
%%   }
%% \else
%%   \newcommand{\todo}[1]{%
%%     \refstepcounter{task}%
%%     {\htmlspan{todo}{#1}}%
%%   }
%%   \newcommand{\bug}[2]{%
%%     \refstepcounter{task}%
%%     {\htmlspan{bug}{#1 #2}}%
%%   }
%% \fi
%% \newcommand\listoftasks\listoftask

%% %
%% % Data structures
%% %

%% \newcommand\listdatastructurename{List of Data Structures} 
%% \newlistof{listofdatastructure}{lods}{\listdatastructurename}
%% \newcounter{datastructure}
%% %\newcommand{\datastructure}[1]{%
%% %  \refstepcounter{datastructure}%
%% %  \par\noindent\textbf{Data Structure \thedatastructure. #1}
%% %  \addcontentsline{lods}{section}{\protect\numberline{\thedatastructure}#1}\par%
%% %}
%% \newcommand\listofdatastructures\listofdatastructure


%% %%%%%%%%%%%%%%%%%%%%%%%%%%%%%%%%%%%%%%%%%%%%%%%%%%%%%%%%%%%%%%%%%%%%%%%%%%%%%%%
%% %%
%% %% LISTINGS
%% %%
%% %%%%%%%%%%%%%%%%%%%%%%%%%%%%%%%%%%%%%%%%%%%%%%%%%%%%%%%%%%%%%%%%%%%%%%%%%%%%%%%

%% \newcommand\lstkbd[1]{%
%%   \ifxetex%
%%     \ensuremath{\mathbf{\textbf{#1}}}%
%%   \else%
%%     \htmlspan{input}{#1 }%
%%   \fi%
%% }
%% %\newcommand\lstfkbd[1]{\underline{\mathbf{\textbf{#1}}}}
%% \newcommand\lstfkbd[1]{\color{cftoutline}{\mathbf{\textbf{#1}}}}
%% \ifxetex
%%   \lstset{%
%%           keywordstyle=\fontspec{Inconsolata Bold},%
%%           keywordstyle=[2]\color{cftoutline}\fontspec{Inconsolata Bold},%
%%           keywordstyle=[3]\fontspec{Inconsolata Bold},%
%%           commentstyle=\color{cftlight}%
%%   }
%% \else
%%   \lstset{%
%%           keywordstyle=\textbf,%
%%           keywordstyle=[2]\textbf,%
%%           keywordstyle=[3]\textit,%
%%           commentstyle=\texttt%
%%   }
%% \fi
%% \lstdefinestyle{deb}{
%%   mathescape=true,
%%   numbers=none,
%%   moredelim=*[s][\textbf]{[}{]}
%% }
%% \lstdefinestyle{forthprogram}{}
%% \lstdefinelanguage{cftasm}{%
%%         mathescape=true,
%%         morekeywords={TRAP,IOT,LOAD,STORE,IN,OUT,JMP,JSR,ADD,AND,OR,%
%%                       XOR,OP1,OP2,ISZ,LIA,R,I,IFL,IFV,CLA,CLL,NOT,%
%%                       INC,CPL,RBL,RBR,RNL,RNR,NOP,SNA,SZA,SSL,SSV,SKIP,%
%%                       SNN,SNZ,SCL,SCV,CLI,SEI,SEL,NEG,ING,LI,SPA,SNP,RET,%
%%                       RTT,RTI,SBL,SBR},%
%%         morekeywords=[2]{.equ,.reg,.include,.word,.fill,%
%%                       .str,.data,.strp,.strn,.page,.macro,.end},%
%%         alsoletter=.,%
%%         sensitive=false,%
%%         morecomment=[l]{/},%
%%         morecomment=[l]{;},%
%% }

%% \lstdefinestyle{longmcasm}{%
%%         language=mcasm,
%%         xleftmargin=25pt,
%%         xrightmargin=5pt,
%%         framexleftmargin=20pt,
%%         basicstyle={\footnotesize\ttfamily},
%% }
%% \lstdefinelanguage{mcasm}{%
%%         mathescape=false,
%%         morekeywords={cond,field,signal,start,hold},%
%%         morekeywords=[2]{\#define,\#ifdef,\#endif,\#if,\#undef,\#line,\#warning,\#warn,\#error},%
%%         morekeywords=[3]{INT,RST,V,L,OP,I,SKIP,INC,uaddr},%
%%         alsoletter=\#,%
%%         sensitive=false,%
%%         morecomment=[l]{//},%
%%         %morecomment=[s]{( }{ )},%
%% }


%% \lstdefinelanguage{forth}{%
%%         mathescape=true,
%%         %morekeywords={TRAP,IOT,LOAD,STORE,IN,OUT,JMP,JSR,ADD,AND,OR,%
%%         %              XOR,OP1,OP2,ISZ,LIA,R,I,IFL,IFV,CLA,CLL,NOT,%
%%         %              INC,CPL,RBL,RBR,RNL,RNR,NOP,SNA,SZA,SSL,SSV,SKIP,%
%%         %              SNN,SNZ,SCL,SCV,CLI,SEI,SEL,NEG,ING,LI,SPA,SNP,RET,%
%%         %              RTT,RTI,SBL,SBR},%
%%         morekeywords=[3]{ok}
%%         %alsoletter=.,%
%%         sensitive=false,%
%%         %morecomment=[l]{\},%
%%         %morecomment=[s]{( }{ )},%
%% }
%% \newcommand\notes[1]{{\small\verbatiminput{#1}}}


%% %%%%%%%%%%%%%%%%%%%%%%%%%%%%%%%%%%%%%%%%%%%%%%%%%%%%%%%%%%%%%%%%%%%%%%%%%%%%%%%
%% %%
%% %% GRAPHICS
%% %%
%% %%%%%%%%%%%%%%%%%%%%%%%%%%%%%%%%%%%%%%%%%%%%%%%%%%%%%%%%%%%%%%%%%%%%%%%%%%%%%%%

%% % This includes a TikZ figure (if compiling with XeTeX), or a static
%% % PNG image (for htLaTeX).
%% \ifxetex
%%   \newcommand{\inputfigure}[2][]{\input{#2}}
%% \else
%%   %% \newcommand{\inputfigure}[2][]{%
%%   %%   \begin{htmldiv}{includegraphics png}%
%%   %%     \includegraphics{#2.png}%
%%   %%   \end{htmldiv}%
%%   %% }
%%   \newcommand{\inputfigure}[2][]{%
%%     \begin{htmldiv}{includegraphics svg}%
%%       %\special{t4ht*<#2.svg}
%%       \HCode{<img src="#2.svg" />}
%%     \end{htmldiv}%
%%   }
%% \fi


%% % This includes a PDF image (for PDF generation), or a PNG
%% % (autoconverted from the PDF).
%% \ifxetex
%%   \newcommand{\includeimage}[2]{\includegraphics[#1]{#2.pdf}}
%% \else
%%   \newcommand{\includeimage}[2][]{%
%%     \begin{htmldiv}{includegraphics png}%
%%       \includegraphics[#1]{#2.png}%
%%     \end{htmldiv}%
%%   }
%% \fi

%% \ifxetex
%%   \newcommand{\includelarge}[2]{\includegraphics[#1]{#2}}
%% \else
%%   \newcommand{\includelarge}[2][]{%
%%     \begin{htmldiv}{includegraphics large jpeg}%
%%       \includegraphics[#1]{#2}%
%%     \end{htmldiv}%
%%   }
%% \fi

%% \ifxetex
%%   \newcommand{\includesmall}[2]{\includegraphics[#1]{#2}}
%% \else
%%   \newcommand{\includesmall}[2][]{%
%%     \begin{htmldiv}{includegraphics small jpeg}%
%%       \includegraphics[#1]{#2}%
%%     \end{htmldiv}%
%%   }
%% \fi

%%%%%%%%%%%%%%%%%%%%%%%%%%%%%%%%%%%%%%%%%%%%%%%%%%%%%%%%%%%%%%%%%%%%%%%%%%%%%%%
%%
%% TYPOGRAPHY
%%
%%%%%%%%%%%%%%%%%%%%%%%%%%%%%%%%%%%%%%%%%%%%%%%%%%%%%%%%%%%%%%%%%%%%%%%%%%%%%%%

\newcommand\textcond{%
  \ifxetex%
    \fontspec{Myriad Pro Condensed}%
  \else%
    \relax%
  \fi%
}


% Hypertext
\newcommand\hyperemail[1]{\sffamily\href{mailto:#1}{#1}}
\newcommand\link[1]{\sffamily\href{http://#1}{#1}}
\newcommand\ahref[2]{\sffamily\href{#1}{#2}}

% Basic stuff
\newcommand\hex[1]{\textsf{#1}}
\newcommand\bin[1]{\textsf{#1}}

% Signals
\newcommand\tU{$\uparrow$}
\newcommand\tD{$\downarrow$}
\ifxetex
  \newcommand{\nsni}[1]{$\overline{\mbox{\textsf{{#1}}}}$}
  \newcommand{\psni}[1]{\textsf{#1}}
\else
  \newcommand{\nsni}[1]{\htmlspan{signal neg}{#1}}
  \newcommand{\psni}[1]{\htmlspan{signal}{#1}}
\fi
\newcommand{\ps}[1]{\index{#1@\psni{#1}}%
  \psni{#1}}
\newcommand{\ns}[1]{\index{#1@{$\protect\overline{\protect\mbox{\textsf{#1}}}$}}%
  \nsni{#1}}
\newcommand\BUS[2]{\ps{#1}$_{\mbox{\scriptsize #2}}$}
\newcommand\nBUS[2]{\ns{#1}$_{\mbox{\scriptsize #2}}$}

% Less than basic stuff
\newcommand{\asm}[1]{\texttt{#1}}
\newcommand{\register}[1]{\textsf{#1}\index{Registers!#1}}
\newcommand{\bus}[1]{{#1}}
\newcommand{\unit}[1]{{#1}}
\newcommand{\board}[1]{#1\index{Boards!#1}}
\newcommand{\lt}[1]{\textsf{#1}}
\newcommand{\sw}[1]{\textsf{#1\index{Switch, front panel!#1}}}
\newcommand{\instr}[1]{\asm{#1}}
\newcommand{\HC}[1]{\chip{74HC{#1}}}
\newcommand{\HCT}[1]{\chip{74HCT{#1}}}
\newcommand{\chip}[1]{#1\index{#1}}
\newcommand{\schpt}[1]{#1\textsf{#1}}
\newcommand\field[1]{\textsf{#1}}
\newcommand\port[1]{\textsf{#1}}
\newcommand\bit[1]{{\texttt{#1}}}

% CFT input and output typesetting
\newcommand\cftin[1]{\textsf{#1}}
\newcommand\cftout[1]{\textsf{#1}}
\let\cftcode\cftout
\let\cftkbd\cftin

% Machine registers
\newcommand\A{\register{AC}}
\newcommand\AC{\A}
\newcommand\DR{\register{DR}}
\newcommand\PC{\register{PC}}
\newcommand\IR{\register{IR}}
\newcommand\AR{\register{AR}}
\newcommand\MAR{\AR}
\newcommand\Areg{\A}
\newcommand\Ireg{\register{I}}
\newcommand\Lreg{\register{L}}
\newcommand\Zreg{\register{Z}}
\newcommand\Vreg{\register{V}}
\newcommand\Nreg{\register{N}}
\newcommand\LAC{\ensuremath{<}\Lreg,~\AC\ensuremath{>}}

% Buses and units
\newcommand\IBUS{\bus{\gls{IBUS}\index{IBUS}}}
\newcommand\DBUS{\bus{\gls{Data Bus}\index{Data Bus}}}
\newcommand\AEXT{\bus{\gls{AExt}\index{AExt}}}
\newcommand\Aext{\bus{\gls{AExt}\index{AExt}}}
\newcommand\ABUS{\bus{\gls{Address Bus}\index{Address Bus}}}
\newcommand\ALU{\unit{\abbr{ALU}\index{ALU}}}

\newcommand\SBU{\unit{\abbr{SBU}\index{SBU}}}
\newcommand\AGL{\unit{\abbr{AGL}\index{AGL}}}
\newcommand\MBU{\unit{\abbr{MBU}\index{MBU}}}

% Signals
\newcommand\CLOCK[1]{\BUS{CLK}{#1}}
\newcommand\CLKn[1]{\CLOCK{#1}}
\newcommand\CLL{\ns{CLL}}
\newcommand\CPL{\ns{CPL}}
\newcommand\STPAC{\ns{STPAC}}
\newcommand\STPDR{\ns{STPDR}}
\newcommand\UINSTR{\ns{uINSTR18}}
\newcommand\HALT{\ns{HALT}}
\newcommand\END{\ns{END}}
\newcommand\IRQ{\ns{IRQ}}
\newcommand\IRQS{\ns{IRQS}}
\newcommand\IRQn[1]{\nBUS{IRQ}{#1}}
\newcommand\RUNITn[1]{\BUS{RUNIT}{#1}}
\newcommand\WUNITn[1]{\BUS{WUNIT}{#1}}
\newcommand\TPA{\ps{TPA}}
\newcommand\TPC{\ps{TPC}}
\newcommand\WAC{\ns{WAC}}
\newcommand\WALU{\ns{WALU}}
\newcommand\WDR{\ns{WDR}}
\newcommand\WIR{\ns{WIR}}
\newcommand\WMAR{\ns{WMAR}}
\newcommand\WPC{\ns{WPC}}
\newcommand\SYSDEV{\ns{SYSDEV}}
\newcommand\IODEV[1]{\ns{IODEV{#1}XX}}
\newcommand\OPIFn[1]{\BUS{OPIF}{#1}}
\newcommand\OPIF{\ps{OPIF}}
\newcommand\GUARDPULSE{\ns{GUARD}}
\newcommand\GP{\GUARDPULSE}
\newcommand\RSTHOLD{\ns{RSTHOLD}}
\newcommand\BOE{\ns{BOE}}
\newcommand\UOE{\ns{UOE}}
\newcommand\SKIP{\ns{SKIP}}
\newcommand\AINDEX{\ps{AINDEX}}
\newcommand\CLI{\ns{CLI}}
\newcommand\STI{\ns{STI}}
\newcommand\IRn[1]{\BUS{IR}{#1}}
\newcommand\PCn[1]{\BUS{PC}{#1}}
\newcommand\IBUSn[1]{\BUS{IBUS}{#1}}
\newcommand\ACn[1]{\BUS{AC}{#1}}
\newcommand\DBUSn[1]{\BUS{DBUS}{#1}}
\newcommand\ABUSn[1]{\BUS{AB}{#1}}
\newcommand\AEXTn[1]{\BUS{AEXT}{#1}}
\newcommand\ISROLL{\ps{ISROLL}}
\newcommand\RAC{\ns{RAC}}
\newcommand\RAGL{\ns{RAGL}}
\newcommand\RDR{\ns{RDR}}
\newcommand\RPC{\ns{RPC}}
\newcommand\INCPC{\ns{INCPC}}
\newcommand\INCAC{\STPAC}
\newcommand\INCDR{\STPDR}
\newcommand\DEC{\ns{DEC}}
\newcommand\MEM{\ns{MEM}}
\newcommand\IO{\ns{IO}}
\newcommand\R{\ns{R}}
\newcommand\WRITE{\ns{W}}
\newcommand\WEN{\ns{WEN}}
\newcommand\WAR{\ns{WAR}}
\newcommand\READ{\ns{R}}
\newcommand\FL{\ps{FL}}
\newcommand\FV{\ps{FV}}
\newcommand\FZERO{\ps{FZERO}}
\newcommand\FNEG{\ps{FNEG}}
\newcommand\RESET{\ns{RESET}}
\newcommand\abbr[1]{#1}
\newcommand\SKIPEXT{\ns{SKIPEXT}}
\newcommand\ENDEXT{\ns{ENDEXT}}
\newcommand\WS{\ns{WS}}
\newcommand\UPC{\ps{µPC}}
\newcommand\UCB{\ps{µCB}}
\newcommand\ACCPL{\ns{ACCPL}}

% Semantics
\newcommand\mem[1]{\mbox{\bfseries mem}\left[#1\right]}
\newcommand\memmem[1]{\mbox{\bfseries mem}\left[\mbox{\bfseries mem}\left[{#1}\right]\right]}
\newcommand\io[1]{\mbox{\bfseries io}\left[#1\right]}
\newcommand\eq{\leftarrow}

% Forth
\newcommand\f[1]{{\texttt{#1}}}



%%%%%%%%%%%%%%%%%%%%%%%%%%%%%%%%%%%%%%%%%%%%%%%%%%%%%%%%%%%%%%%%%%%%%%%%%%%%%%%
%%
%% MEMORY LOCATIONS
%%
%%%%%%%%%%%%%%%%%%%%%%%%%%%%%%%%%%%%%%%%%%%%%%%%%%%%%%%%%%%%%%%%%%%%%%%%%%%%%%%

\makeatletter
\newcommand\ioport@[4]{%
  \label{ioport:#1-#4}
  \vspace{0.5em}
  \noindent\hex{\bfseries{#2}} (\texttt{#1}): {\bfseries\asm{\bfseries{#3}}} — {#4}
  \vspace{0.5em}
}

% \ioport{port}{crwvehf}{regname}{descr}
%
%% \newcommand\ioport[4]{%
%%   \ioport@{#1}{#2}{#3}{#4}
%%   \addcontentsline{loioport}{section}{\hex{#2} (\texttt{#1}) \textbf{\asm{#3}} — soup}%
%% }

\newenvironment{ioport}[5]{%
  \begin{htmldiv}{ioport}
    \vspace{0.5em}
    \addcontentsline{loioport}{section}{\hex{#2} (\texttt{#3}) \textbf{\cftout{#1} \cftout{#4} — #5}}%
    \noindent\hex{\bfseries{#2}} (\texttt{#3}): {\bfseries\asm{\bfseries{#4}}} — {#5}%
    \noindent%
}{%
    \vspace{0.5em}
  \end{htmldiv}
}

\ifxetex
  \newenvironment{extcmd}[7]{%
    \vspace{0.5em}
    %% \addcontentsline{loioport}{section}{\hex{#2} (\texttt{#4}) \textbf{\cftout{#1} \cftout{#2} — #6}}%
    \noindent\hex{\bfseries{#2}} \hex{#3} (I/O port \hex{#4} — \texttt{#5}): {\bfseries{#6}}%
    \noindent%
  }{%
    \vspace{0.5em}
  }
\else
  \newenvironment{extcmd}[6]{%
    \begin{htmldiv}{extcmd}
      \begin{htmldiv}{header}
        %% \addcontentsline{loioport}{section}{\hex{#2} (\texttt{#4}) \textbf{\cftout{#1} \cftout{#2} — #6}}%
        \noindent%
        \htmlspan{instruction}{#2}
        \htmlspan{addr}{I/O port \hex{#3}}
        \htmlspan{flags}{#4}
        \htmlspan{title}{#6}
      \end{htmldiv}
      \begin{htmldiv}{body}
  }{%
      \end{htmldiv}
    \end{htmldiv}
  }
\fi


% \extcmda{HALT}{OUT R 000A}{540A}{crwvehf}{00a}{Short Descr}{Long Descr}
\newcommand\extcmda[7]{%
  \begin{htmldiv}{extcmda}
    \label{ioport:#5-#2}
    \vspace{0.5em}
    \noindent\hex{\bfseries{#2}} (\texttt{#1}): {\bfseries\asm{\bfseries{#3}}} — {#4}
    \vspace{0.5em}
    #1 #2 #3 #4 #5 #6 #7
    %% \ioport{#4}{#5}{#1}{#7}
  \end{htmldiv}
}
\makeatother


%%%%%%%%%%%%%%%%%%%%%%%%%%%%%%%%%%%%%%%%%%%%%%%%%%%%%%%%%%%%%%%%%%%%%%%%%%%%%%%
%%
%% TABLES
%%
%%%%%%%%%%%%%%%%%%%%%%%%%%%%%%%%%%%%%%%%%%%%%%%%%%%%%%%%%%%%%%%%%%%%%%%%%%%%%%%

\ifxetex
  \newcommand{\zebrarow}[1]{\rowcolors{#1}{cfthl!50}{cfthl!25}}
  \newcommand\zebra{\zebrarow{2}}
  \newcommand\zebrahdr{\zebrarow{1}}
  %\newcommand\zebra*[1]{\rowcolors{#1}{gray!10}{white}}
\else
  \newcommand{\zebrarow}[1]{}
  \newcommand\zebra{\zebrarow{2}}
  \newcommand\zebrahdr{\zebrarow{1}}
\fi


%%%%%%%%%%%%%%%%%%%%%%%%%%%%%%%%%%%%%%%%%%%%%%%%%%%%%%%%%%%%%%%%%%%%%%%%%%%%%%%
%
% DISPLAYING AND INDEXING SCHEMATICS
%
%%%%%%%%%%%%%%%%%%%%%%%%%%%%%%%%%%%%%%%%%%%%%%%%%%%%%%%%%%%%%%%%%%%%%%%%%%%%%%%


% \schematic{page number}{description}{label}
\newcounter{schematic}
\def\schematicsFile{figs/schematics}
\newcommand\includesch[3]{%
  \stepcounter{subsection}%
  \phantomsection%
  \addcontentsline{toc}{subsection}{\protect\numberline{\thesubsection} #2}%
  \includeschns{#1}{#2}{#3}
}

\newcommand\includeschns[3]{%
  \label{#3}%
  \stepcounter{schematic}%
  \addcontentsline{los}{section}{\protect\numberline{\theschematic} #2}%
  \ifxetex
    \includepdf[
      pages={#1}
      ,landscape,
      ,fitpaper=true,
  %    ,pagecommand={\thispagestyle{lscape}}  
      ,pagecommand={\thispagestyle{empty}}  
    ]{\schematicsFile.pdf}%
  \else
    \begin{htmldiv}{includegraphics large landscape schematic}
      \includegraphics{\schematicsFile-#1.png}%
    \end{htmldiv}
  \fi
}


%%%%%%%%%%%%%%%%%%%%%%%%%%%%%%%%%%%%%%%%%%%%%%%%%%%%%%%%%%%%%%%%%%%%%%%%%%%%%%%
%%
%% BITFIELDS
%%
%%%%%%%%%%%%%%%%%%%%%%%%%%%%%%%%%%%%%%%%%%%%%%%%%%%%%%%%%%%%%%%%%%%%%%%%%%%%%%%

\newcounter{bitfieldBit}
\makeatletter
\def\bitfieldHeight{0.7}
\def\bitfieldHeightText{0.225}
\def\bitfieldTickMark{0.15}
\def\bitfieldBits{16}
\ifxetex
\else
  \newcommand\bitfield@cells{}
\fi

\newenvironment{bitfield@}[1][]{%
  \ifxetex
    \pgfmathsetmacro{\bitfieldBitsMinusOne}{\bitfieldBits - 1}
    \pgfmathsetmacro{\bitfieldBitsMinusTwo}{\bitfieldBits - 2}
    \pgfmathsetmacro{\bitfieldStep}{\bitfieldWidth / \bitfieldBits}
    \vspace{0.5em}
    \setcounter{bitfieldBit}{0}
    \begin{tikzpicture}
      \draw[fill=white, heavy] (0,0) rectangle (\bitfieldWidth,\bitfieldHeight);
      \foreach \x in {0, ..., \bitfieldBitsMinusOne}{
        \begin{scope}[xshift=\bitfieldWidth cm - \bitfieldStep * (\x cm + 1 cm)]
          \draw(0,0) -- +(0,\bitfieldHeight);
          \draw(\bitfieldStep / 2, \bitfieldHeight / 2) %
          node {\textcond\small\textbf{#1}};
          \draw[color=cftdark!50](\bitfieldStep / 2,\bitfieldHeight) node[above] {\scriptsize\x};
        \end{scope}
        \foreach \x in {0, ..., \bitfieldBitsMinusTwo}{
          \draw[xshift=\bitfieldWidth cm - \bitfieldStep * (\x cm + 1 cm)]%
          (0,\bitfieldHeight) -- +(0, \bitfieldTickMark);
        }
      }
  \else
    \HCode{<div class="bitfield"><table><thead><tr class="bitnumbers">}
    % Print out the bit numbers
    \count@\bitfieldBits
    \loop\ifnum\count@>0
      \advance\count@-1
      \HCode{<th class="b\the\count@">\the\count@</th>}
    \repeat
    % Print out the tick marks
    \HCode{</tr><tr class="ticks">}
    \count@\bitfieldBits
    \loop\ifnum\count@>0
      \advance\count@-1
      \HCode{<th></th>}
    \repeat
    \HCode{</tr></thead><tbody><tr class="fields">}
    % Bits come in the reverse order of what HTML tables expect, so
    % store the cells in a LIFO macro.
    \def\bitfield@cells{}
  \fi
}{%
  \ifxetex
    \draw[fill=none, heavy] (0,0) rectangle (\bitfieldWidth,\bitfieldHeight);
    \end{tikzpicture}
    \vspace{0.5em}
  \else
    \bitfield@cells
    \HCode{</tr></tbody></table></div>}
  \fi
}
\newenvironment{bitfield}[1][]{%
  \def\bitfieldWidth{10}
  \begin{center}%
    \begin{bitfield@}{#1}%
}{%
    \end{bitfield@}
  \end{center}
}
\newenvironment{cbitfield}[1][]{%
  \def\bitfieldWidth{14}
  \begin{center}%
    \begin{bitfield@}{#1}%
}{%
    \end{bitfield@}
  \end{center}
}

\newenvironment{nbitfield}[2][]{%
  \def\bitfieldWidth{14}
  \def\bitfieldBits{#2}
  \begin{center}%
    \begin{bitfield@}{#1}%
}{%
    \end{bitfield@}
  \end{center}
}

\newcommand\bitfieldItem[3]{%
  \ifxetex
    \begin{scope}[xshift=\bitfieldWidth cm - \bitfieldStep * (\arabic{bitfieldBit} cm + #1 cm)]
      \draw[fill=#2, draw opacity=0] (0,0) rectangle (\bitfieldStep * #1, \bitfieldHeight);
      \draw(\bitfieldStep * #1 / 2, \bitfieldHeightText) %
      node[anchor=base] {\textcond{\small {#3}}};
      \draw[thick] (0,0) -- +(0, \bitfieldHeight);
      \draw[thick,xshift=\bitfieldStep * #1 cm] (0,0) -- +(0, \bitfieldHeight);
    \end{scope}
    \addtocounter{bitfieldBit}{#1}
  \else
    \edef\bitfield@cells{\HCode{<td colspan="#1" data-color="#2" class="condensed">#3</td>}\bitfield@cells}
  \fi
}

\newcommand\bitfieldGroup[3]{%
  \ifxetex
    \begin{scope}[xshift=\bitfieldWidth cm - \bitfieldStep * (\arabic{bitfieldBit} cm + #1 cm)]
      \draw[fill=#2, draw opacity=0] (0,0) rectangle (\bitfieldStep * #1, \bitfieldHeight);
      \draw(\bitfieldStep * #1 / 2, \bitfieldHeightText) %
      node[anchor=base] {\textcond{\small{#3}} };
      \draw[thick] (0,0) -- +(0, \bitfieldHeight);
      \draw[heavy,xshift=\bitfieldStep * #1 cm] (0,0) -- +(0, \bitfieldHeight);
    \end{scope}
    \addtocounter{bitfieldBit}{#1}
  \else 
    \edef\bitfield@cells{\HCode{<td data-bg="#2" class="condensed" colspan="#1">#3</td>}\bitfield@cells}
  \fi
}

\newcommand\bitfieldConst[1]{%
  \ifxetex
    \bitfieldItem{1}{white}{\textbf{#1}}
  \else
    \edef\bitfield@cells{\HCode{<td class="constant">#1</td>}\bitfield@cells}
  \fi
}
\newcommand\bitfieldRepConst[2]{%
  \ifxetex
    \foreach \x in {1,...,#1} \bitfieldConst{#2};
    \draw[heavy,xshift=\bitfieldStep * #1 cm] (0,0) -- +(0, \bitfieldHeight);
  \else
    \count@#1
    \loop\ifnum\count@>0
      \advance\count@-1
      \edef\bitfield@cells{\HCode{<td class="condensed constant">#2</td>}\bitfield@cells}
    \repeat
  \fi
}

\makeatother



%%%%%%%%%%%%%%%%%%%%%%%%%%%%%%%%%%%%%%%%%%%%%%%%%%%%%%%%%%%%%%%%%%%%%%%%%%%%%%%
%
% DATA STRUCTURES
%
%%%%%%%%%%%%%%%%%%%%%%%%%%%%%%%%%%%%%%%%%%%%%%%%%%%%%%%%%%%%%%%%%%%%%%%%%%%%%%%

% Data structures
\newcommand\ds[1]{{\ttfamily #1\index{#1@{\texttt{#1}}}}}
\makeatletter
\newcommand{\simpledatastructure}[1]{%
  \label{ds:#1}
  \refstepcounter{datastructure}%
  \addcontentsline{lods}{section}{\protect\numberline{\thedatastructure}{\ttfamily #1}}%
  \index{#1@{\texttt{#1}}|(pie}%
  {\textbf{\texttt{#1}:}}
}
\newenvironment{datastructure}[2][Address]{%
  \refstepcounter{datastructure}%
  \addcontentsline{lods}{section}{\protect\numberline{\thedatastructure}{\ttfamily #2}}%
  \index{#2@{\texttt{#2}}|(pie}%
  \label{ds:#1}%
  \begin{htmldiv}{datastructure}%
    \begin{center}%
      \zebrarow{10}%
      \begin{longtable}{>{\textbf\bgroup}r<{\egroup}lp{.7\columnwidth}}
        %
        % First header
        %
        \hiderowcolors
        {#1} & Type & Description\\
        \hline
        \noalign{\global\rownum 0\relax}\showrowcolors
        \endfirsthead
        %
        % Subsequent headers
        %
        \hiderowcolors
        \multicolumn{3}{l}{\em Continued from previous page.}\\
        \noalign{\smallskip\smallskip}
        {#1} & Type & Description\\
        \hline
        \noalign{\global\rownum 1\relax}\showrowcolors
        \endhead
        %
        % Footer
        %
        \hiderowcolors
        \hline\noalign{\smallskip\smallskip}
        \multicolumn{3}{r}{\em Continued on next page.}\\
        \endfoot
        %
        % Last footer
        %
        \hiderowcolors
        \hline
        \endlastfoot
        %
        % Content
        %
        \showrowcolors
}{%
      \end{longtable}
    \end{center}%
  \end{htmldiv}%
  \@afterindentfalse%
  \@afterheading%
}
\makeatother
\newcommand\dsdesc[3]{
{#1}&\ds{#2}&{#3}\\
}


%%%%%%%%%%%%%%%%%%%%%%%%%%%%%%%%%%%%%%%%%%%%%%%%%%%%%%%%%%%%%%%%%%%%%%%%%%%%%%%
%%
%% CONVENIENCE
%%
%%%%%%%%%%%%%%%%%%%%%%%%%%%%%%%%%%%%%%%%%%%%%%%%%%%%%%%%%%%%%%%%%%%%%%%%%%%%%%%

\newcommand\op[1]{{\ttfamily #1}}
\newcommand\fwni[1]{{\ttfamily{#1}}}
\newcommand\fw[1]{\fwni{#1}\index{#1@\protect\fwni{#1}}}
%                           \index{#1@\fwni{#1}|(pie}%




\newcommand\li[1]{\item[\bfseries #1]}
\newcommand\NB{\textbf{Nota Bene:\ }}



%% % End of file.
