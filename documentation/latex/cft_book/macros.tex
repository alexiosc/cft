% -*- latex -*-


%%%%%%%%%%%%%%%%%%%%%%%%%%%%%%%%%%%%%%%%%%%%%%%%%%%%%%%%%%%%%%%%%%%%%%%%%%%%%%%
%%
%% WORKAROUNDS
%%
%%%%%%%%%%%%%%%%%%%%%%%%%%%%%%%%%%%%%%%%%%%%%%%%%%%%%%%%%%%%%%%%%%%%%%%%%%%%%%%

\ifxetex
  % XeTeX doesn't have HTML output, of course.
  \def\HCode#1{}
\else
  % This is a hack.
  \newcounter{Hfootnote}
\fi



%%%%%%%%%%%%%%%%%%%%%%%%%%%%%%%%%%%%%%%%%%%%%%%%%%%%%%%%%%%%%%%%%%%%%%%%%%%%%%%
%%
%% HTML GENERATION
%%
%%%%%%%%%%%%%%%%%%%%%%%%%%%%%%%%%%%%%%%%%%%%%%%%%%%%%%%%%%%%%%%%%%%%%%%%%%%%%%%

\ifxetex
  \def\texonly#1{#1}
  \def\htmlonly#1{}
  \newenvironment{htmldiv}[1]{}{}
  \newcommand{\htmlspan}[2]{#2}
\else
  \def\texonly#1{}
  \def\htmlonly#1{#1}
  \newenvironment{htmldiv}[1]{\HCode{<div class="#1">}}{\HCode{</div>}}
  \newcommand{\htmlspan}[2]{\HCode{<span class="#1">}{#2}\HCode{</span>}}
\fi


%%%%%%%%%%%%%%%%%%%%%%%%%%%%%%%%%%%%%%%%%%%%%%%%%%%%%%%%%%%%%%%%%%%%%%%%%%%%%%%
%%
%% LINKING TO LOCATIONS IN THE DOCUMENT
%%
%%%%%%%%%%%%%%%%%%%%%%%%%%%%%%%%%%%%%%%%%%%%%%%%%%%%%%%%%%%%%%%%%%%%%%%%%%%%%%%

% Use hyperlinking when rendering PDFs
\newcommand{\barecf}[1]{\hyperref[#1]{\ref*{#1}}}

\newcommand{\cf}[2][section]{\hyperref[#2]{%
        \ifthenelse{\equal{\pageref*{#2}}{\thepage}}%
        {#1 \ref*{#2}}%
        {#1 \ref*{#2} (p.~\pageref*{#2})}%
}}

\newcommand{\cfp}[2][section]{\hyperref[#2]{%
        \ifthenelse{\equal{\pageref*{#2}}{\thepage}}%
        {#1 \ref*{#2}}%
        {#1 \ref*{#2}, p.~\pageref*{#2}}%
}}

\newcommand{\fcf}[1]{\cf[figure]{#1}}
\newcommand{\fcfp}[1]{\cfp[figure]{#1}}
\newcommand{\tcf}[1]{\cf[table]{#1}}
\newcommand{\tcfp}[1]{\cfp[table]{#1}}
\newcommand{\ccf}[1]{\cf[chapter]{#1}}
\newcommand{\ccfp}[1]{\cfp[chapter]{#1}}
%
\newcommand{\npcf}[2][section]{\hyperref[#2]{#1 \ref*{#2}}}
\newcommand{\appcf}[1]{\cf[appendix]{#1}}
\newcommand{\ecf}[1]{\cf[equation]{#1}}
\newcommand{\algcf}[1]{\cf[algorithm]{#1}}
\newcommand{\npappcf}[1]{\npcf[appendix]{#1}}
\newcommand{\npccf}[1]{\npcf[chapter]{#1}}
\newcommand{\npfcf}[1]{\npcf[figure]{#1}}
\newcommand{\nptcf}[1]{\npcf[table]{#1}}
\newcommand{\npecf}[1]{\npcf[equation]{#1}}
\newcommand{\npalgcf}[1]{\npcf[algorithm]{#1}}



%%%%%%%%%%%%%%%%%%%%%%%%%%%%%%%%%%%%%%%%%%%%%%%%%%%%%%%%%%%%%%%%%%%%%%%%%%%%%%%
%%
%% LISTS OF THINGS
%%
%%%%%%%%%%%%%%%%%%%%%%%%%%%%%%%%%%%%%%%%%%%%%%%%%%%%%%%%%%%%%%%%%%%%%%%%%%%%%%%

%% %%%\addtolength\cftfignumwidth{1.5em}
\ifxetex
  \makeatletter
  \renewcommand*\l@section{\@dottedtocline{1}{1.5em}{2.3em}}
  \renewcommand*\l@subsection{\@dottedtocline{2}{3.8em}{3.2em}}
  \renewcommand*\l@subsubsection{\@dottedtocline{3}{7.0em}{4.1em}}
  \renewcommand*\l@paragraph{\@dottedtocline{4}{10em}{5em}}
  \renewcommand*\l@subparagraph{\@dottedtocline{5}{12em}{6em}}
  \setcounter{maxsecnumdepth}{3}

  \renewcommand{\@pnumwidth}{3em}
  \renewcommand{\@tocrmarg}{4em}
  \makeatother
\else
  \relax
\fi

%
% Schematics
%

\newcommand\listschematicname{List of Schematics} 
\newcommand{\schematic}[1]{%
  \refstepcounter{schematic}%
  \par\noindent\textbf{Schematic \theschematic. #1}
  \addcontentsline{los}{section}{\protect\numberline{\theschematic}#1}\par%
}
\newcommand{\listofschematics}{\listofschematic}

%
% I/O Ports
%

\newcommand\listioportname{List of Input/Output Ports} 
\newlistof{listofioport}{loioport}{\listioportname}
%% \newcommand{\registerioport}[1]{%
%%   \refstepcounter{ioport}%
%%   \addcontentsline{loioport}{section}{\protect\numberline{\ }%
%% }
\newcommand\listofioports\listofioport



\newcommand\caution[1]{\textcolor{caution}{\textbf{#1}}}
%\newcommand\todo[1]{\textcolor{caution}{\bf{TODO: #1}}}

%
% Tasks
%

\newcommand\listtasksname{List of Incomplete Tasks} 
\newlistof{listoftask}{lotasks}{\listtasksname}
\newcounter{task}

\ifxetex
  \newcommand{\todo}[1]{%
    \refstepcounter{task}%
    {\textcolor{caution}{\textbf{TODO: #1}}}
    \addcontentsline{lotasks}{section}{\protect\numberline{\arabic{task}}To Do: #1}\par%
  }
  \newcommand{\bug}[2]{%
    \refstepcounter{task}%
    {\textcolor{caution}{\textbf{BUG: #1 #2}}}
    \addcontentsline{lotasks}{section}{\protect\numberline{\arabic{task}}Bug: #1}\par%
  }
\else
  \newcommand{\todo}[1]{%
    \refstepcounter{task}%
    {\htmlspan{todo}{#1}}%
  }
  \newcommand{\bug}[2]{%
    \refstepcounter{task}%
    {\htmlspan{bug}{#1 #2}}%
  }
\fi
\newcommand\listoftasks\listoftask

%
% Data structures
%

\newcommand\listdatastructurename{List of Data Structures} 
\newlistof{listofdatastructure}{lods}{\listdatastructurename}
\newcounter{datastructure}
%\newcommand{\datastructure}[1]{%
%  \refstepcounter{datastructure}%
%  \par\noindent\textbf{Data Structure \thedatastructure. #1}
%  \addcontentsline{lods}{section}{\protect\numberline{\thedatastructure}#1}\par%
%}
\newcommand\listofdatastructures\listofdatastructure


%%%%%%%%%%%%%%%%%%%%%%%%%%%%%%%%%%%%%%%%%%%%%%%%%%%%%%%%%%%%%%%%%%%%%%%%%%%%%%%
%%
%% LISTINGS
%%
%%%%%%%%%%%%%%%%%%%%%%%%%%%%%%%%%%%%%%%%%%%%%%%%%%%%%%%%%%%%%%%%%%%%%%%%%%%%%%%

\newcommand\lstkbd[1]{\mathbf{\textbf{#1}}}
%\newcommand\lstfkbd[1]{\underline{\mathbf{\textbf{#1}}}}
\newcommand\lstfkbd[1]{\color{cftoutline}{\mathbf{\textbf{#1}}}}
\ifxetex
  \lstset{%
          keywordstyle=\fontspec{Inconsolata Bold},%
          keywordstyle=[2]\color{cftoutline}\fontspec{Inconsolata Bold},%
          keywordstyle=[3]\fontspec{Inconsolata Bold},%
          commentstyle=\color{cftlight}%
  }
\else
  \lstset{%
          keywordstyle=\textbf,%
          keywordstyle=[2]\textbf,%
          keywordstyle=[3]\textit,%
          commentstyle=\texttt%
  }
\fi
\lstdefinestyle{deb}{mathescape=true,numbers=none}
\lstdefinestyle{forthprogram}{}
\lstdefinelanguage{cftasm}{%
        mathescape=true,
        morekeywords={TRAP,IOT,LOAD,STORE,IN,OUT,JMP,JSR,ADD,AND,OR,%
                      XOR,OP1,OP2,ISZ,LIA,R,I,IFL,IFV,CLA,CLL,NOT,%
                      INC,CPL,RBL,RBR,RNL,RNR,NOP,SNA,SZA,SSL,SSV,SKIP,%
                      SNN,SNZ,SCL,SCV,CLI,SEI,SEL,NEG,ING,LI,SPA,SNP,RET,%
                      RTT,RTI,SBL,SBR},%
        morekeywords=[2]{.equ,.reg,.include,.word,.fill,%
                      .str,.data,.strp,.strn,.page,.macro,.end},%
        alsoletter=.,%
        sensitive=false,%
        morecomment=[l]{/},%
        morecomment=[l]{;},%
}

\lstdefinestyle{longmcasm}{%
        language=mcasm,
        xleftmargin=25pt,
        xrightmargin=5pt,
        framexleftmargin=20pt,
        basicstyle={\footnotesize\ttfamily},
}
\lstdefinelanguage{mcasm}{%
        mathescape=false,
        morekeywords={cond,field,signal,start,hold},%
        morekeywords=[2]{\#define,\#ifdef,\#endif,\#if,\#undef,\#line,\#warning,\#warn,\#error},%
        morekeywords=[3]{INT,RST,V,L,OP,I,SKIP,INC,uaddr},%
        alsoletter=\#,%
        sensitive=false,%
        morecomment=[l]{//},%
        %morecomment=[s]{( }{ )},%
}


\lstdefinelanguage{forth}{%
        mathescape=true,
        %morekeywords={TRAP,IOT,LOAD,STORE,IN,OUT,JMP,JSR,ADD,AND,OR,%
        %              XOR,OP1,OP2,ISZ,LIA,R,I,IFL,IFV,CLA,CLL,NOT,%
        %              INC,CPL,RBL,RBR,RNL,RNR,NOP,SNA,SZA,SSL,SSV,SKIP,%
        %              SNN,SNZ,SCL,SCV,CLI,SEI,SEL,NEG,ING,LI,SPA,SNP,RET,%
        %              RTT,RTI,SBL,SBR},%
        morekeywords=[3]{ok}
        %alsoletter=.,%
        sensitive=false,%
        %morecomment=[l]{\},%
        %morecomment=[s]{( }{ )},%
}


%%%%%%%%%%%%%%%%%%%%%%%%%%%%%%%%%%%%%%%%%%%%%%%%%%%%%%%%%%%%%%%%%%%%%%%%%%%%%%%
%%
%% GRAPHICS
%%
%%%%%%%%%%%%%%%%%%%%%%%%%%%%%%%%%%%%%%%%%%%%%%%%%%%%%%%%%%%%%%%%%%%%%%%%%%%%%%%

% This includes a TikZ figure (if compiling with XeTeX), or a static
% PNG image (for htLaTeX).
\ifxetex
  \newcommand{\inputfigure}[1]{\input{#1}}
\else
  \newcommand{\inputfigure}[1]{%
    \begin{htmldiv}{includegraphics}%
      \includegraphics{#1.png}%
    \end{htmldiv}%
  }
\fi


%%%%%%%%%%%%%%%%%%%%%%%%%%%%%%%%%%%%%%%%%%%%%%%%%%%%%%%%%%%%%%%%%%%%%%%%%%%%%%%
%%
%% TYPOGRAPHY
%%
%%%%%%%%%%%%%%%%%%%%%%%%%%%%%%%%%%%%%%%%%%%%%%%%%%%%%%%%%%%%%%%%%%%%%%%%%%%%%%%

% Hypertext
\newcommand\hyperemail[1]{\sffamily\href{mailto:#1}{#1}}
\newcommand\link[1]{\sffamily\href{http://#1}{#1}}
\newcommand\ahref[2]{\sffamily\href{#1}{#2}}

% Basic stuff
\newcommand\hex[1]{\textsf{#1}}
\newcommand\bin[1]{\textsf{#1}}

% Signals
\newcommand\tU{$\uparrow$}
\newcommand\tD{$\downarrow$}
\ifxetex
  \newcommand{\nsni}[1]{$\overline{\mbox{\textsf{{#1}}}}$}
  \newcommand{\psni}[1]{\textsf{#1}}
\else
  \newcommand{\nsni}[1]{\htmlspan{signal neg}{#1}}
  \newcommand{\psni}[1]{\htmlspan{signal}{#1}}
\fi
\newcommand{\ps}[1]{\index{#1@\psni{#1}}%
  \psni{#1}}
\newcommand{\ns}[1]{\index{#1@{$\protect\overline{\protect\mbox{\textsf{#1}}}$}}%
  \nsni{#1}}
\newcommand\BUS[2]{\ps{#1}$_{\mbox{\scriptsize #2}}$}
\newcommand\nBUS[2]{\ns{#1}$_{\mbox{\scriptsize #2}}$}

% Less than basic stuff
\newcommand{\asm}[1]{\texttt{#1}}
\newcommand\register[1]{\textsf{#1}\index{Registers!#1}}
\newcommand\bus[1]{{#1}}
\newcommand\unit[1]{{#1}}
\newcommand\board[1]{#1\index{Boards!#1}}


% CFT input and output typesetting
\newcommand\cftin[1]{\textsf{#1}}
\newcommand\cftout[1]{\textsf{#1}}
\let\cftcode\cftout
\let\cftkbd\cftin

% Machine registers
\newcommand\A{\register{AC}}
\newcommand\AC{\A}
\newcommand\DR{\register{DR}}
\newcommand\PC{\register{PC}}
\newcommand\IR{\register{IR}}
\newcommand\AR{\register{AR}}
\newcommand\MAR{\AR}
\newcommand\Areg{\A}
\newcommand\Ireg{\register{I}}
\newcommand\Lreg{\register{L}}
\newcommand\Zreg{\register{Z}}
\newcommand\Vreg{\register{V}}
\newcommand\Nreg{\register{N}}

% Buses and units
\newcommand\IBUS{\bus{\gls{IBUS}\index{IBUS}}}
\newcommand\DBUS{\bus{\gls{Data Bus}\index{Data Bus}}}
\newcommand\AEXT{\bus{\gls{AExt}\index{AExt}}}
\newcommand\ABUS{\bus{\gls{Address Bus}\index{Address Bus}}}
\newcommand\ALU{\unit{\gls{ALU}\index{ALU}}}

\newcommand\SBU{\unit{\gls{SBU}\index{SBU}}}
\newcommand\AGL{\unit{\gls{AGL}\index{AGL}}}

% Signals
\newcommand\CLOCK[1]{\BUS{CLK}{#1}}
\newcommand\CLKn[1]{\CLOCK{#1}}
\newcommand\CLL{\ns{CLL}}
\newcommand\CPL{\ns{CPL}}
\newcommand\STPAC{\ns{STPAC}}
\newcommand\STPDR{\ns{STPDR}}
\newcommand\UINSTR{\ns{uINSTR18}}
\newcommand\HALT{\ns{HALT}}
\newcommand\END{\ns{END}}
\newcommand\IRQ{\ns{IRQ}}
\newcommand\IRQS{\ns{IRQS}}
\newcommand\IRQn[1]{\nBUS{IRQ}{#1}}
\newcommand\RUNITn[1]{\BUS{RUNIT}{#1}}
\newcommand\WUNITn[1]{\BUS{WUNIT}{#1}}
\newcommand\TPA{\ps{TPA}}
\newcommand\TPC{\ps{TPC}}
\newcommand\WAC{\ns{WAC}}
\newcommand\WALU{\ns{WALU}}
\newcommand\WDR{\ns{WDR}}
\newcommand\WIR{\ns{WIR}}
\newcommand\WMAR{\ns{WMAR}}
\newcommand\WPC{\ns{WPC}}
\newcommand\SYSDEV{\ns{SYSDEV}}
\newcommand\IODEV[1]{\ns{IODEV{#1}XX}}
\newcommand\OPIFn[1]{\BUS{OPIF}{#1}}
\newcommand\OPIF{\ps{OPIF}}
\newcommand\GUARDPULSE{\ns{GUARD}}
\newcommand\GP{\GUARDPULSE}
\newcommand\RSTHOLD{\ns{RSTHOLD}}
\newcommand\BOE{\ns{BOE}}
\newcommand\UOE{\ns{UOE}}
\newcommand\SKIP{\ns{SKIP}}
\newcommand\AINDEX{\ps{AINDEX}}
\newcommand\CLI{\ns{CLI}}
\newcommand\STI{\ns{STI}}
\newcommand\IRn[1]{\BUS{IR}{#1}}
\newcommand\PCn[1]{\BUS{PC}{#1}}
\newcommand\IBUSn[1]{\BUS{IBUS}{#1}}
\newcommand\ACn[1]{\BUS{AC}{#1}}
\newcommand\DBUSn[1]{\BUS{DBUS}{#1}}
\newcommand\ABUSn[1]{\BUS{AB}{#1}}
\newcommand\AEXTn[1]{\BUS{AEXT}{#1}}
\newcommand\ISROLL{\ps{ISROLL}}
\newcommand\RAC{\ns{RAC}}
\newcommand\RAGL{\ns{RAGL}}
\newcommand\RDR{\ns{RDR}}
\newcommand\RPC{\ns{RPC}}
\newcommand\INCPC{\ns{INCPC}}
\newcommand\INCAC{\STPAC}
\newcommand\INCDR{\STPDR}
\newcommand\DEC{\ns{DEC}}
\newcommand\MEM{\ns{MEM}}
\newcommand\IO{\ns{IO}}
\newcommand\R{\ns{R}}
\newcommand\WRITE{\ns{W}}
\newcommand\WEN{\ns{WEN}}
\newcommand\WAR{\ns{WAR}}
\newcommand\READ{\ns{R}}
\newcommand\FL{\ps{FL}}
\newcommand\FV{\ps{FV}}
\newcommand\FZERO{\ps{FZERO}}
\newcommand\FNEG{\ps{FNEG}}
\newcommand\RESET{\ns{RESET}}
\newcommand\abbr[1]{#1}
\newcommand\SKIPEXT{\ns{SKIPEXT}}
\newcommand\ENDEXT{\ns{ENDEXT}}
\newcommand\WS{\ns{WS}}
\newcommand\UPC{\ps{µPC}}
\newcommand\UCB{\ps{µCB}}
\newcommand\ACCPL{\ns{ACCPL}}


%%%%%%%%%%%%%%%%%%%%%%%%%%%%%%%%%%%%%%%%%%%%%%%%%%%%%%%%%%%%%%%%%%%%%%%%%%%%%%%
%%
%% MEMORY LOCATIONS
%%
%%%%%%%%%%%%%%%%%%%%%%%%%%%%%%%%%%%%%%%%%%%%%%%%%%%%%%%%%%%%%%%%%%%%%%%%%%%%%%%

\makeatletter
\newcommand\ioport@[4]{%
  \label{ioport:#1-#4}
  \vspace{0.5em}
  \noindent\hex{\bfseries{#2}} (\texttt{#1}): {\bfseries\asm{\bfseries{#3}}} — {#4}
  \vspace{0.5em}
}

% \ioport{port}{crwvehf}{regname}{descr}
%
%% \newcommand\ioport[4]{%
%%   \ioport@{#1}{#2}{#3}{#4}
%%   \addcontentsline{loioport}{section}{\hex{#2} (\texttt{#1}) \textbf{\asm{#3}} — soup}%
%% }

\newenvironment{ioport}[5]{%
  \begin{htmldiv}{ioport}
    \vspace{0.5em}
    \addcontentsline{loioport}{section}{\hex{#2} (\texttt{#3}) \textbf{\cftout{#1} \cftout{#4} — #5}}%
    \noindent\hex{\bfseries{#2}} (\texttt{#3}): {\bfseries\asm{\bfseries{#4}}} — {#5}%
    \noindent%
}{%
  \vspace{0.5em}
  \end{htmldiv}
}

\newenvironment{extcmd}[7]{%
  \begin{htmldiv}{extcmd}
    \vspace{0.5em}
    \addcontentsline{loioport}{section}{\hex{#2} (\texttt{#4}) \textbf{\cftout{#1} \cftout{#2} — #6}}%
    \noindent\hex{\bfseries{#2}} \hex{#3} (I/O port \hex{#4} — \texttt{#5}): {\bfseries{#6}}%
    \noindent%
}{%
  \vspace{0.5em}
  \end{htmldiv}
}


% \extcmda{HALT}{OUT R 000A}{540A}{crwvehf}{00a}{Short Descr}{Long Descr}
\newcommand\extcmda[7]{%
  \begin{htmldiv}{extcmda}
    \label{ioport:#5-#2}
    \vspace{0.5em}
    \noindent\hex{\bfseries{#2}} (\texttt{#1}): {\bfseries\asm{\bfseries{#3}}} — {#4}
    \vspace{0.5em}
    #1 #2 #3 #4 #5 #6 #7
    %% \ioport{#4}{#5}{#1}{#7}
  \end{htmldiv}
}
\makeatother


%%%%%%%%%%%%%%%%%%%%%%%%%%%%%%%%%%%%%%%%%%%%%%%%%%%%%%%%%%%%%%%%%%%%%%%%%%%%%%%
%%
%% TABLES
%%
%%%%%%%%%%%%%%%%%%%%%%%%%%%%%%%%%%%%%%%%%%%%%%%%%%%%%%%%%%%%%%%%%%%%%%%%%%%%%%%

\ifxetex
  \newcommand{\zebrarow}[1]{\rowcolors{#1}{cfthl!50}{cfthl!25}}
  \newcommand\zebra{\zebrarow{2}}
  \newcommand\zebrahdr{\zebrarow{1}}
  %\newcommand\zebra*[1]{\rowcolors{#1}{gray!10}{white}}
\else
  \newcommand{\zebrarow}[1]{}
  \newcommand\zebra{\zebrarow{2}}
  \newcommand\zebrahdr{\zebrarow{1}}
\fi


%%%%%%%%%%%%%%%%%%%%%%%%%%%%%%%%%%%%%%%%%%%%%%%%%%%%%%%%%%%%%%%%%%%%%%%%%%%%%%%
%%
%% CONVENIENCE
%%
%%%%%%%%%%%%%%%%%%%%%%%%%%%%%%%%%%%%%%%%%%%%%%%%%%%%%%%%%%%%%%%%%%%%%%%%%%%%%%%

\newcommand\op[1]{{\ttfamily #1}}
\newcommand\fwni[1]{{\ttfamily{#1}}}
\newcommand\fw[1]{\fwni{#1}\index{#1@\protect\fwni{#1}}}
%                           \index{#1@\fwni{#1}|(pie}%




\newcommand\li[1]{\item{\bfseries #1}}
\newcommand\NB{\textbf{Nota Bene:\ }}



% End of file.
