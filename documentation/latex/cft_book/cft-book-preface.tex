\chapter{Typographical Conventions}

Because of the breadth of the project, a wide number of typographical
conventions are followed in the book. In some cases, different entities may not
be distinguishable.

\section{Basic Conventions}

\begin{description}

\item {\em Italics\/} is used for emphasis.

\item \textbf{Boldface} is used for more emphasis, and to highlight descriptions
  in lists.

\item \hex{80:53FE} Hexadecimal numbers are set like this. Where necessary, a
  colon (\hex{:}) is used to group hexadecimal numbers in 16-bit (four digit)
  quantities.

\item \bin{0101:00110101} Binary numbers are shown like this. In some cases,
  colons (\bin{:}) are used to break long binary numbers at appropriate places
  that depend on what is being discussed.

\item \asm{LOAD R \&10} Inside text, Assembly instructions are shown like this.

\item \AC{} Processor registers are shown like this.

\item \cftout{Ready.} Output from the CFT is shown like this.

\item \cftin{WORDS} Input to the CFT (commands typed by the user) is shown like
  this.

\item \hyperemail{alexios@bedroomlan.org} Email addresses are typeset like
  this. They are hypertext links. You can follow them by clicking or tapping to
  trigger them. This will usually allow you to send an email to that address.

\item \link{www.bedroomlan.org/cft} World-Wide Web addresses are typeset like
  this. They are hypertext links. You can follow them to the linked page by
  clicking or tapping.

\end{description}

\section{Program Listings and Computer Interaction}

Listings are shown like this. Listings longer than a couple of lines are
line-numbered, with the numbers appearing left of the listing.

\begin{lstlisting}[language=cftasm]
; Comments are set like this.

LOAD R 10    ; Keywords appear in bold.
.word &0F00  ; Directives appear like this.

prompt> $\lstkbd{User input is shown like this.}$
\end{lstlisting}

\noindent Forth interactions are quite compact, and user input and machine
responses often share the same line. The Forth user input convention is always
used to identify user input in this case, and is made more prominent by
underlining:

\begin{lstlisting}[language=forth]
Ready.
$\lstfkbd{15 3 * .}$ 45  ok
$\lstfkbd{foo}$ FOO ?  ok
\end{lstlisting}

\section{Logic}

In describing logic, the following conventions are used:

\begin{description}

\item \ps{RESET} Active-high signal names are shown like this.

\item \ns{HALT} Active-low signals are set like active-high signals, with an overbar.

\item \bin{1} or \textsf{H} (depending on context) is used for a
  signal's high level (which is almost always +5V with respect to
  ground).

\item \bin{0} or \textsf{L} (depending on context) is used for a
  signal's low level (which is always at +0V with respect to ground).

\item \textsf{X} denotes a ‘don't-care’ value, denoting either a high
  or low level. This is used in function tables to show cases where an
  input is ignored.

\item \tU{} is the rising edge of a signal. It is used to denote a
  circuit's sensitivity to a signal edge (event), not a level (state).

\item \tD{} is the falling edge of a signal. It is used to denote a
  circuit's sensitivity to a signal edge (event), not a level (state).

\end{description}

\section{I/O Addresses and Extended Commands}

%% \makeatletter
%% \ioport@{crwvehf}{315}{IOREG}{%
%%   I/O addresses are shown like this.
%%   %
%%   \begin{description}
%%   \item {\texttt{\textbf{c}}} This address is used as a command or instruction.
%%   \item {\texttt{\textbf{r}}} This address can be read from.
%%   \item {\texttt{\textbf{w}}} This address can be written to.
%%   \item {\texttt{\textbf{v}}} This is available in the Verilog verification version.
%%   \item {\texttt{\textbf{e}}} This is available in the CFT Emulator.
%%   \item {\texttt{\textbf{h}}} This is available on the hardware version of the CFT.
%%   \item {\texttt{\textbf{f}}} This address is fully decoded, i.e. only
%%     one copy of the instruction or device is available in the address space.
%%   \end{description}
%% }
%% \makeatother

%% \extcmda{HALT}{OUT R \&a}{540a}{crwvehf}{00a}%
%%         {PFP, Halt Processor}%
%%         {%
%%          Extended CFT instructions are shown like this. Extended instructions
%%          are in essence magic I/O space addresses that perform a particular
%%          task when read from or written to. In this case, the address is
%%          \hex{00a}. The instruction is dubbed \asm{HALT}. The letters in the
%%          parentheses signify the same as for I/O addresses above.
%%        }

% End of file.
