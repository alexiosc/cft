% -*- latex -*-
\HtmlMetaDescription{The Interrupt Controller board (IRC) provides
  eight levels of interrupt requests to help with asynchronous device
  I/O.}
%\HtmlMetaGoogleDescription{}
%\HtmlMetaBanner{}
%\HtmlMetaTags{}


The Interrupt Controller Board (IRC) eases the load of interrupt handling on
the CFT by receiving up to eight interrupt lines from peripherals, aggregating
them into the CFT processor's single \ns{IRQ} line, and allowing the source of
an interrupt to be determined. Interrupts not needed may be masked as required.

\section{Design}

The CFT has a single, maskable interrupt request line. While this is sufficient
for simple uses, as the number of peripherals grows, load-balancing interrupt
traffic is useful: it helps keep the \gls{ISR} short, which reduces the chance
of losing interrupts (typical CFT ISRs are non-reentrant, and interrupts are
disabled while they execute).

The Interrupt Controller Card works as a register and multiplexer of up to
eight interrupt request lines from other cards. Unlike interrupt controllers
for other platforms, this one is a very simple device. It lacks interrupt
prioritisation, and all interrupts are always edge-triggered. Any
prioritisation of interrupt servicing is coded entirely in software.

The Interrupt Controller simply provides a central repository for interrupt
events. The ISR can mask interrupts to disable unwanted ones, and it can query
the Interrupt Controller to determine which lines have had interrupt
activity. This allows the ISR to perform one eighth of the work.

Electrically, sharing of interrupts is possible, since all interrupt lines are
open-drain ones. Logically, as long as devices drive an interrupt line low for
a sufficiently short time, sharing is possible. Any peripherals that hold the
interrupt line low until the interrupt is acknowledged with them specifically
will cause trouble in a sharing environment. Some of this trouble can be dealt
with in software and using the time-honoured art of interrupt line
juggling. Astable vibrators or edge detectors can also be placed between such
devices and their interrupt lines to simplify handling. This is mainly expected
from devices made for level-triggered interrupts such as \glspl{UART}.

Even though the IRC card's output is open-drain, it is not meant to be shared
with other interrupt sources. The card will keep the \ns{IRQ} line asserted
(driven low) until all pending interrupts are acknowledged by the host.

\section{Theory of Operation}

The centre of the interrupt controller is a group of eight flip-flops (each
half of a \HC{112} J-K flip-flop chip). Each J-K flip-flop handles one
interrupt request line.

Each flip-flop has its J (set) signal tied high, and its K (clear) signal tied
low. The \ns{IRQn} line is connected to the CLK signal. On the falling edge of
\ns{IRQn}, the flip-flop is set, registering an interrupt request.

To mask interrupts, an enable bit, \ps{IEN0}–\ps{IEN7} is connected to the
asynchronous, active low \ns{CLR} signal of the flip-flop. When the enable
signal is clear (low), the flip-flop remains in a reset state, ignoring clock
pulses. When the enable signal goes high, the flip-flop's reset condition
clears and it may respond to incoming interrupts.

The active-high (\ps{Q}) outputs of all eight flip-flops are disjuncted using a
group of eight open-drain NOT gates with their outputs connected to form a
wired AND, whih works as a negative logic wired OR. This in turn drives the
processor's \ns{IRQ} input. Since it is an open drain connection, it only
drives low, allowing other devices to also drive the \ns{IRQ} lin. This is done
for safety, but is unnecessary in a sane system. The presence of the interrupt
controller also implies that all peripherals will be using the separate
\ns{IRQ0}–\ns{IRQ7} lines instead, and the interrupt controller will be the
only device driving the \ns{IRQ} line.

The flip-flops' \ps{Q} outputs are also brought to a \HC{541} buffer, which
drives the lower 8 bits of the Data Bus when required. This is the Interrupt
Status Register, allowing the host to determine the source or sources of an
interrupt.

To control the interrupts, a \HC{259} addressable latch is used. The latch
allows interrupt enable bits to be set individually. The advantage of this over
a simpler eight bit latch or flip-flop is programmatic simplicity: it
simplifies enabling interrupts incrementally, so that each device driver in an
operating system can enable its own interrupt without disabling (or having to
take into account) the interrupt settings of other subsystems. One disadvantage
of this choice is that it now takes sixteen instructions to disable {\em all\/}
interrupts, but this operation is not expected to happen a lot, especially
since the \ns{RESET} signal from the processor also clears the \HC{259}
performing the very same action.

Finally, addressing the card on the CFT bus is performed using a single
\HC{138} demultiplexer, which provides partial decoding: reading from I/O space
addresses \hex{30}–\hex{3F} always read from the Interrupt Status Register,
while writing to any of these addresses writes to the Interrupt Enable
Register.

\section{Implementation}

\todo{IRC Implementation}

\section{Connectors}

The eight interrupt lines used by the Interrupt Controller Board are part of
the CFT Bus, but two connectors for optional front-panel LEDs are provided. The
connectors are meant for the right hand bank of the front panel.

\section{Operation}

\begin{ioport}{IRC}{30–3F}{--w-eh-}{IER}{Interrupt Enable Command Register}

  This write-only register controls which interrupts are allowed to reach the
  processor, and also acknowledges interrupt requests. Due to partial decoding,
  there are 16 copies of this register, \hex{30}–\hex{3F}.

  \begin{cbitfield}
    \bitfieldGroup{1}{cfthl!50}{IEN}
    \bitfieldGroup{3}{cfthl!25}{IRQ}
    \bitfieldRepConst{12}{0}
  \end{cbitfield}

  \begin{description}
    \li{\field{IEN}} This bit selects one of two commands:
    \begin{description}
      \li{\bin{0}}: mask (disable) an interrupt line, clearing its status flag. This
      acknowledges that the host is aware of the interrupt.
      \li{\bin{1}}: unmask (enable) an interrupt line, clearing its status flag. This
      acknowledges that the host is aware of the interrupt.
    \end{description}

    \li{\field{IRQ}} The interrupt line to enable, in the range \hex{0}–\hex{7}.
  \end{description}

  To enable an interrupt request, the number of the interrupt line is shifted
  left by one bit, and one is added before writing to this register. To disable
  an interrupt request, the number of the interrupt line is shifted left by one
  bit, and the value is written to the register. Assembly macros to perform
  these tasks (with an IRQ number as a literal value) are as follows:

\begin{lstlisting}[language=cftasm]
; Enable an IRQ provided as a literal value
.macro IRQEN (irq)
        LI @%irq*2 1            ; (IRQ level << 1) | 1.
        OUT R &30               ; Enable it.
.end

; Disable an IRQ provided as a literal value
.macro IRQDIS (irq)
        LI @%irq*2              ; IRQ level << 1
        OUT R &30               ; Disable it.
.end

; Acknowledge an IRQ provided as a literal value
.macro IRQACK (irq)
        IRQDIS(%irq             ; First disable it
        IRQEN(%irq)             ; Then enable it again
.end
\end{lstlisting}

The macros perform the IRQ number shifting at assembly time. If the IRQ number
is in a register, the \asm{SBL} macro-instruction should be used instead, and
the calculation happens at runtime.
  
\end{ioport}

\begin{ioport}{IRC}{30—3F}{-r--eh-}{ISR}{Interrupt Status Register}

  This read-only register allows the processor to check the status of the eight
  interrupt request lines.

  \begin{cbitfield}
    \bitfieldGroup{1}{cfthl!50}{IRF0}
    \bitfieldGroup{1}{cfthl!25}{IRF1}
    \bitfieldGroup{1}{cfthl!50}{IRF2}
    \bitfieldGroup{1}{cfthl!25}{IRF3}
    \bitfieldGroup{1}{cfthl!50}{IRF4}
    \bitfieldGroup{1}{cfthl!25}{IRF5}
    \bitfieldGroup{1}{cfthl!50}{IRF6}
    \bitfieldGroup{1}{cfthl!25}{IRF7}
    \bitfieldRepConst{8}{0}
  \end{cbitfield}

  Each of these eight bits \field{IRF0}–\field{IRF1} represents an interrupt
  line from \ns{IRQ0} to \ns{IRQ7}. A set bit (\bin{1}) indicates that an
  interrupt request has been received. A clear bit (\bin{0}) indicates no
  interrupt activity on that line.

\end{ioport}

\section{Schematics}

The following page shows the full schematic drawings of the Interrupt Controller board.

\cleardoublepage
\includeschns{22}{Interrupt Controller Board}{sch2:irc}
