% -*- latex -*-

\renewcommand\glspostdescription{}

\newcommand\newglossaryabbr[3]{%
  \newglossaryentry{abbr#1}{name=\glslink{#1}{#2}, text={#2}, description={#3}}
  \newacronym[description={\glslink{abbr#1}{#2}}]{#1}{#1}{#2}
}

\newglossaryentry{IBUS}{name=IBUS,
  description={
  Internal Bus: the single, internal 16-bit bus of the CFT processor.
  }}
  
\newglossaryentry{Data Bus}{name=Data Bus, description={ A
    16-bit bus used to move data between the processor and
    peripherals.  }}
  
\newglossaryentry{Address Bus}{name=Address Bus, description={A
    16-bit bus used to select a peripheral nor memory location to
    access.}}
  
\newglossaryentry{Video Display Unit}{name=Video Display Unit,
  description={ Video Display Unit (VDU): a device that drives a
    display monitor to show character or graphics data on a
    screen. Often also contains one or more input devices (keyboard,
    mouse).}}
  
\newglossaryentry{von Neumann architecture}{name=Von Neumann architecture,
  description={ Named after the work of
    John von Neumann, the von Neumann architecture is a ‘modern’
    computer architecture that includes a register file, arithmetic
    and logic units, memory and input/output, and uses the same
    storage for programs and data.}}

\newglossaryentry{stored program computer}{name=Stored program computer,
  description={ A type of computer that
    uses the same storage for programs and data. All modern computers
    are stored program computers, allowing self-modification of their
    programs. Early designs and many modern microcontrollers stored
    programs and data in separate media, and did not or could not
    treat programs as data.}}

\newglossaryabbr{ALU}{Arithmetic/Logic Unit}{Describe this!}
\newglossaryabbr{ISR}{Interrupt Service Routine}{Describe this!}
\newglossaryentry{UART}{name=UART, description={
    Short for Universal Asynchronous Receiver/Transmitter. UARTs are
    devices that handle asynchronous serial communications such as RS-232, RS-485 or USB.
    }
}
%% \newglossaryentry{alu}{name=\glslink{ALU}{Arithmetic/Logic Unit}, text=Arithmetic/Logic Unit,
%%   description={TODO}}
%% \newacronym[description={\glslink{alu}{Arithmetic/Logic Unit}}]{ALU}{ALU}{Arithmetic/Logic Unit}

\newacronym{SBU}{SBU}{Skip/Branch Unit}

\newacronym{AGL}{AGL}{Address Generation Logic}

\newglossaryentry{DEB}{name=DEB, description={
    Designation of the CFT Debugging/Testing board, a peripheral that
    allows remote control and testing of the computer and its
    peripherals in the style of a virtual front panel. The DEB board
    is discussed in~\ccf{chap:deb}.
}}

\newglossaryentry{VDU}{name=VDU, description={ Video Display
    Unit. Designation of the CFT graphic card and keyboard
    controller. The VDU board is discussed in~\ccf{chap:vdu}.  }}

\newglossaryentry{MSB}{name=MSB, description={
    Most Significant Bits. Usually used to refer to the upper eight bits of a CFT word.
}}
\newglossaryentry{LSB}{name=LSB, description={
    Least Significant Bits. Usually used to refer to the lower eight bits of a CFT word.
}}

\newglossaryentry{NYBBLE}{name=Nybble, description={Sometimes
    nibble. A four-bit quantity, represented as a single hexadecimal
    digit.
}}

% \newacronym{MCU}{MCU}{Micro-Controller Unit}

\newglossaryabbr{MCU}{Micro-Controller Unit}{A single-chip device
  containing a simple microprocessor, general purpose input/output,
  serial input/output, timers, and other useful devices.}

\newacronym{USB}{USB}{Universal Serial Bus}

\newacronym{GPIO}{GPIO}{General-Purpose Input/Output}

\newglossaryentry{Verilog}{name=Verilog,
  description={
    One of the two major hardware
    description languages, the other being VHDL. In the CFT project,
    Verilog is used to perform 74xxx chip-level simulation and
    verification of the CFT processor.
}}

\newglossaryentry{I2C}{name=I²C, description={A two-wire, open-drain
    bus, often used for communication between \gls{MCU}s and
    peripheral chips, including EEPROMs and sensors. For more
    information, please consult
    \url{http://en.wikipedia.org/wiki/I2C}.}}

\newglossaryentry{data stack}{name=Data Stack,
  description=To Do}

\newglossaryentry{postfix}{name=postfix,
  description=To Do}

\newglossaryentry{stack effect comment}{name=stack effect comment,
  description=To Do}

\newglossaryentry{Wait State}{name=Wait State, 
  description={An additional state added to the processor state machine
  to accommodate slow devices. External devices signal their need for a
  wait state to the processor using an appropriate procotol. The
  processor then enters the Wait State and does not leave until the
  device is ready. The processor performs no action during this time,
  hence the name.}}


\newglossaryentry{Twos Complement}{name=Two's Complement,
  description={Currently the most popular means of representing signed
    integers in binary. For a bit width $n$, non-negative numbers $0 \leq
    x < 2^{n-1}$ are represented as unsigned $n$-bit binary
    integers. Negative numbers $-2^{n-1} \leq x < 0$ are represented in
    the form $2^n - x$, with the most significant bit set. On the CFT,
    with $n=16$, this provides a signed range of $[-32,768,
      32,767]$. Two's complement has many useful mathematical
    properties: only one representation of zero, the most significant
    bit also acts as a sign bit, and both addition and subtraction can
    be performed by the same circuitry. For a full discussion, please refer
    to \url{http://en.wikipedia.org/wiki/Two's\_complement}.  }}

\newglossaryentry{Assembly}{name=Assembly, description={A symbolic form of
    \gls{machine code}, meant to be used by humans. Some architectures,
    including the CFT, have simple machine code that can be learned in minutes,
    but the human brain deals better with symbols. Assembly languages also
    provide productivity enhancing features like comments, labels, macros,
    simple arithmetic for literals, and other such facilities. The CFT
    Assembler provides all of these facilities.  }}

\newglossaryentry{DIP}{name=DIP, description={Dual In-line Package, the
    most common integrated circuit package until the introduction of
    Surface Mount Technology: DIP chips have two rows of pins, with pins
    set at distances of 2.54~mm (0.1~inch).
}}

\newglossaryentry{PLCC}{name=PLCC, description={Plastic Leaded Chip
    Carrier, a square surface mount chip packaging easily adapted to
    2.54mm grids by means of suitable IC sockets. The CFT uses
    numerous PLCC chips to save board space and because they offer
    more choice of chips, and thus better value for money. All CFT
    PLCC packages are either Flash devices or UARTs.}}

\newglossaryentry{MEM}{name=MEM, description={ Designation of the
    memory board, which, depending on construction details provides up
    to 512 or 1,024 kWords of RAM and/or ROM. The MEM board is
    discussed in~\ccf{chap:mem}.
}}

\newglossaryentry{MBU}{name=MBU, description={Designation of the
    Memory Banking Unit. This was originally slated to be a separate
    peripheral, but has become so important to the project that it is
    now considered to be part of the processor, and constructed as
    such. The MBU breaks memory space in logical and physical
    addresses, and allows the processor's limited 64 kWord address
    space to be expanded to a 21 bit, 2,048 kWord address space using
    memory banking techniques. More details may be found in
    \ccf{chap:mbu}.  }}

\newglossaryentry{Interrupt Service Routine}{name=Interrupt Service Routine (ISR),
  description={TO DO}}

\newglossaryentry{Page}{name=Page, description={The CFT instruction set allows
    for a 10-bit operand, which allows instructions to address up to
    $2^{10}=1024$ locations. This 1,024-word range is known as a page. The CFT
    address space contains 64 such 1 kWord pages. \gls{Page Zero} at addresses
    \hex{0000}–\hex{03FF} has special significance in the programming model, as
    discussed in~\cf{sec:memory-space}.
 }}

\newglossaryentry{Page Zero}{name=Page Zero, description={Page Zero occupies
    the first 1,024 addresses (\hex{0000}–\hex{03FFF}) of the memory address
    space. It is used to simulate 1,024 registers, global variables or
    constants accessible from anywhere in memory. Addresses
    \hex{0080}–\hex{00FF} are autoindex locations: they increment when accessed
    in \gls{Indirect Mode}.}}

\newglossaryentry{Addressing Mode}{name=Addressing Mode, description={An
    addressing mode is the method in which an instruction operand is
    interpreted. The CFT can treat operands as literal values (\gls{Literal
      Mode}), addresses of data (\gls{Direct Mode}), or addresses of addresses
    of data (\gls{Indirect Mode}), similar to pointersin high-level languages
    such as C and Pascal. Addressing modes are discussed
    in~\cf{sec:addressing-modes}}.}

\newglossaryentry{Indirect Mode}{name=Indirect Mode, description={An addressing
    mode where an instruction operand identifies the memory location which
    contains the address of the data to access. Compare \gls{Direct
      Mode}. Addressing modes are discussed in~\cf{sec:addressing-modes}.}}

\newglossaryentry{Direct Mode}{name=Direct Mode, description={An addressing
    mode where an instruction operand identifies the memory location of the
    data to access. Compare \gls{Indirect Mode}. Addressing modes are discussed
    in~\cf{sec:addressing-modes}.}}

\newglossaryentry{Literal Mode}{name=Literal Mode, description={An addressing
    mode where an instruction operand identifies the memory location of the
    data to access. Compare \gls{Indirect Mode}. Addressing modes are discussed
    in~\cf{sec:addressing-modes}.}}

\newglossaryentry{register}{name=Register, description={ A data storage device
    inside a processor. Registers are usually faster than memory, and so are
    used to store intermediate results. In many architectures, notably
    \glspl{ABA}, the only way for the processor to
    manipulate data is to transfer it to a register, operate on the register,
    then transfer it back to its intended location. CFT registers are discussed
    in~\cf{sec:registers}.}}

\newglossaryentry{Accumulator}{name=Accumulator, description={The main
    \gls{register} in an \gls{ABA}. Data must be transferred to the Accumulator
    before it can be operated on by the processor. In the CFT, the Accumulator
    is the only such general purpose register available. The term comes from
    the days of tabulating machines, which used accumulators to accumulate
    (sum) numbers. The Accumulator is designated ‘\AC’ on the CFT. Its hardware
    design is discussed in~\cf{sec:major-registers} and its operation from a
    programmer's point of view is discussed in~\cf{sec:accumulator}.}}

\newglossaryabbr{ABA}{Accumulator-Based Architecture}{A processor architecture
  built around a single (or in some cases, a few) accumulators. Most early
  computers followed this design because of its simplicity, and the CFT does
  too. Having a single register draastically simplifies the instruction format,
  too: from a two-operand instruction format (source, destination), we move to
  a single operand (source or destination). The other operand is always the
  accumulator.  }

\newglossaryentry{nybble}{name=Nybble, description={(sometimes nibble)
    A 4-bit quantity, corresponding to one hexadecimal digit, half a
    byte, or one quarter of a CFT \gls{Word}.}}

\newglossaryentry{Word}{name=Word, description={One CFT Word is a
    16-bit quantity. CFT memory and I/O accesses transfer exactly one
    word each. CFT instructions are also one Word wide. This is not to
    be confused with the Forth concept of a \gls{word} (an identifier).}}

\newglossaryentry{word}{name=Word, description={A Forth word is any
    sequence of non-whitespace characters that is not a number. This
    is not to be confused with the more common concept of the
    \gls{Word} as a numeric datatype.}}

\newglossaryentry{machine code}{name=Machine code, description={Machine code is
    the native language of every processor, where machine instructions and data
    are represented in binary. Machine code is easy for the computer to
    process, but humans find it useful to apply abstraction layers to it: data
    are represented in other bases (octal, decimal and hexadecimal being the
    most common), and symbolic instruction names (rather than binary
    instruction numbers) are used. This set of abstractions is \gls{Assembly}
    language.}}

\newglossaryentry{disk label}{name=Disk Label, description={ A disk label
    stores information about a storage device (not always a disk), including a
    magic number to help detect the block, some optional boot code, and
    definitions for one to sixteen \glspl{disk slice} (partitions). The disk
    label always resides in the first block of a device. The best-known disk
    label format is the MS-DOS Master Boot Record, MBR. The CFT's disk label is
    discussed in~\cf{sec:disk-label}.}}

\newglossaryentry{disk slice}{name=Disk Slice, description={Part of a disk intended to
    hold a \gls{filesystem} or other data. Disks are sliced to make them easier
    to manage, so different operating systems can be used, to control the size
    of stored data, and to avoid corruption in one filesystem destroying all
    data. In the MS-DOS world, slices are known as ‘partitions’, and the
    \gls{disk label} is known as a ‘partition table’ or ‘Master Boot Record’
    depending whether it resides on a data disk or bootable disk. The CFT's
    slice scheme is discussed in~\cf{sec:disk-slices}}.}

\newglossaryentry{filesystem}{name=Filesystem, description={ A large data
    structure used to view a storage medium as a hierarchical collection of
    data objects called files. The filesystem abstracts the natural
    array-of-blocks structure of the storage medium, and provides the user with
    an interface for creating, reading and otherwise operating on files by
    their name, location and other attributes. Files can be larger than the
    natural block size, and this is again handled transparently by the
    filesystem code. The CFT's filesystem is described in~\cf{sec:fs}}}

\newglossaryabbr{VD}{Volume Directory}{A data structure
  describing a CFT \gls{filesystem}. Comparable to a Unix filesystem
  superblock. The VDD is the exact same data structure as a
  \gls{DD}. The difference is in the magic number and the contents of
  the first directory slot (the header), which contains a volume
  header structure.}

\newglossaryabbr{DD}{Directory Descriptor}{A data structure describing
  a directory in a CFT \gls{filesystem}.}

\newglossaryentry{block pointer}{name=Block Pointer, description={A
    32-bit value (\gls{MSB} first) denoting the number of a filesystem
    block. A filesystem's first block is block \hex{00000000}.}}

\newglossaryabbr{DH}{Directory Header}{A data structure describing a
  CFT directory. This is the first entry of a \gls{DD}.}

\newglossaryabbr{OOP}{Object Oriented Programming}{A programming
  paradigm where data and code are bundled together in data structures
  called objects. Programs are built based on the interactions and
  interrelations of these objects.}

\newglossaryentry{descriptor}{name=Descriptor, description={In the CFT Filesystem, a data structure
    that holds metadata on an object in the CFT filesystem. Descriptors share
    some common metadata (e.g. filenames, flags, creation time and date). They
    are located inside container objects (volumes and directories). Containers
    and descriptors are discussed in~\cfp{sec:fs-containers}.}}

\newglossaryentry{volume}{name=Volume, description={A \gls{disk slice} which
    has been structured according to the CFT Filesystem. In \gls{OOP} terms, a
    volume is the object (instance) and the CFT Filesystem is the class.}}

\newglossaryentry{container}{name=Container, description={In the CFT
    Filesystem, a data structure that contains other filesystem objects. Entire
    filesystem \glspl{volume} are containers, and so are directories. Each
    container owns a maximum number of \glspl{descriptor}, identifying the
    contained objects. Once this limit is reached, continuation blocks are
    allocated for the container. These continuation blocks form a doubly linked
    list. Containers are discussed in~\fcf{sec:fs-containers}.}}

\newglossaryabbr{PEL}{Picture Element}{The smallest addressable
  picture element of a display mode. A PEL is usually a block of
  multiple pixels.}

\newglossaryabbr{MTBF}{Mean Time Between Failures}{The average life
  expectancy of a device.}

\newglossaryentry{codepoint}{name=Codepoint, description={A number
    identifying a character within a character set. For example, in
    the ASCII set, codepoint 65 is the character ‘A’.}}

\newglossaryabbr{SDRAM}{Synchronous Dynamic Random Access Memory}{A
  type of modern dynamic RAM that operates using a clock, not
  asynchronously like original DRAMs. The memory contains a command
  pipeline, because data appears on the bus two to three clock ticks
  after a read is signalled. Multiple reads are pipelined, making the
  memory very fast for many use patterns.}

\newglossaryabbr{SRAM}{Static Random Access Memory}{A type of
  relatively low-density memory that trades off capacity and price for
  speed and interface simplicity. SRAMs do not need to be refreshed
  and are faster than similar dynamic RAMs.}

\newglossaryentry{extended instruction}{name=Instruction!Extended,
  description={An address in I/O space that provides side effects
    useful enough to be seen as extending the processor's instruction
    set. They are usually aliases of single \asm{IN}, \asm{OUT} or
    \asm{IOT} instructions.}}

\newglossaryabbr{SR}{Switch Register}{A 16-bit read-only register that
  provides the value of the 16 data entry switches on the front
  panel.}

\newglossaryabbr{DSR}{DIP Switch Register}{A 12 to 16-bit read-only
  register driven by a bank of DIP switches on the Front Panel
  Controller board. The DSR can be used to set non-volatile prefrences
  for the computer's early boot.}

\newglossaryabbr{MFD}{Multi-Function Display}{A bank of 16 lights on
  the front panel which can display a user-selectable register. The
  user can elect to show the value of the Data Register, Output
  Register, or the 15-bit microcode store address. A switch on the
  front panel is used to do this.}
