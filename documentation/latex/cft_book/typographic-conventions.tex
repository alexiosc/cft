% -*- latex -*-

\definecolor{caution}{RGB}{192,0,0}

\newcommand\textcond{\fontspec{Myriad Pro Condensed}}

% Primary index entries
\newcommand\pie[1]{\textbf{\hyperpage{#1}}}

% Use hyperlinking when rendering PDFs
\newcommand{\barecf}[1]{\hyperref[#1]{\ref*{#1}}}
\newcommand{\cf}[2][section]{\hyperref[#2]{%
        \ifthenelse{\equal{\pageref*{#2}}{\thepage}}%
        {#1 \ref*{#2}}%
        {#1 \ref*{#2} (p.~\pageref*{#2})}%
}}
\newcommand{\cfp}[2][section]{\hyperref[#2]{%
        \ifthenelse{\equal{\pageref*{#2}}{\thepage}}%
        {#1 \ref*{#2}}%
        {#1 \ref*{#2}, p.~\pageref*{#2}}%
}}
\newcommand{\fcf}[1]{\cf[figure]{#1}}
\newcommand{\fcfp}[1]{\cfp[figure]{#1}}
\newcommand{\tcf}[1]{\cf[table]{#1}}
\newcommand{\tcfp}[1]{\cfp[table]{#1}}
\newcommand{\ccf}[1]{\cf[chapter]{#1}}
\newcommand{\ccfp}[1]{\cfp[chapter]{#1}}
%
\newcommand{\npcf}[2][section]{\hyperref[#2]{#1 \ref*{#2}}}
\newcommand{\appcf}[1]{\cf[appendix]{#1}}
\newcommand{\ecf}[1]{\cf[equation]{#1}}
\newcommand{\algcf}[1]{\cf[algorithm]{#1}}
\newcommand{\npappcf}[1]{\npcf[appendix]{#1}}
\newcommand{\npccf}[1]{\npcf[chapter]{#1}}
\newcommand{\npfcf}[1]{\npcf[figure]{#1}}
\newcommand{\nptcf}[1]{\npcf[table]{#1}}
\newcommand{\npecf}[1]{\npcf[equation]{#1}}
\newcommand{\npalgcf}[1]{\npcf[algorithm]{#1}}


\newcommand\f[1]{{\texttt{#1}}}
\newcommand\bitmap[1]{\texttt{#1}}

\newcommand{\instr}[1]{\asm{#1}}
\newcommand{\lt}[1]{\textsf{#1}}
\newcommand{\sw}[1]{\textsf{#1\index{Switch, front panel!#1}}}
\newcommand{\HC}[1]{\chip{74HC{#1}}}
\newcommand{\HCT}[1]{\chip{74HCT{#1}}}
\newcommand{\chip}[1]{#1\index{#1}}

\newcommand{\schpt}[1]{#1\textsf{#1}}
\newcommand{\farnell}[1]{#1}



\newcommand\bit[1]{{\texttt{#1}}}

\newcommand\notes[1]{{\small\verbatiminput{#1}}}

\newcommand\field[1]{\textsf{#1}}
\newcommand\port[1]{\textsf{#1}}


% Temporary question environment
\newcommand\question[2]{\textbf{#1} #2}


%%%%%%%%%%%%%%%%%%%%%%%%%%%%%%%%%%%%%%%%%%%%%%%%%%%%%%%%%%%%%%%%%%%%%%%%%%%%%%%
%
% THE I/O PORT AND EXTENDED COMMAND INDEX
%
%%%%%%%%%%%%%%%%%%%%%%%%%%%%%%%%%%%%%%%%%%%%%%%%%%%%%%%%%%%%%%%%%%%%%%%%%%%%%%%

%% \makeatletter
%% \newcommand\ioport@[4]{%
%%   \label{ioport:#1-#4}
%%   \vspace{0.5em}
%%   \noindent\hex{\bfseries{#2}} (\texttt{#1}): {\bfseries\asm{\bfseries{#3}}} — {#4}
%%   \vspace{0.5em}
%% }
%
%
% \ioport{port}{crwvehf}{regname}{descr}
%
%% \newcommand\ioport[4]{%
%%   \ioport@{#1}{#2}{#3}{#4}
%%   \addcontentsline{loioport}{section}{\hex{#2} (\texttt{#1}) \textbf{\asm{#3}} — soup}%
%% }

% \begin{ioport}{VDU}{1F0}{--wvehf}{MCR0}{Mode Control Register 0}
% ...
% \end{ioport}


% \begin{extcmd}{PFP}{SR1}{4409}{009}{--wvehf}{Mode Control Register 0}
% ...
% \end{extcmd}


%%%%%%%%%%%%%%%%%%%%%%%%%%%%%%%%%%%%%%%%%%%%%%%%%%%%%%%%%%%%%%%%%%%%%%%%%%%%%%%
%
% CODE FOR DISPLAYING AND INDEXING SCHEMATICS
%
%%%%%%%%%%%%%%%%%%%%%%%%%%%%%%%%%%%%%%%%%%%%%%%%%%%%%%%%%%%%%%%%%%%%%%%%%%%%%%%


% \schematic{page number}{description}{label}
\def\schematicsFile{figs/schematics.pdf}
\newcommand\includesch[3]{%
  \stepcounter{subsection}%
  \phantomsection%
  \addcontentsline{toc}{subsection}{\protect\numberline{\thesubsection} #2}%
  \includeschns{#1}{#2}{#3}
}

\newcommand\includeschns[3]{%
  \label{#3}%
  \stepcounter{schematic}%
  \addcontentsline{los}{section}{\protect\numberline{\theschematic} #2}%
  \includepdf[
    pages={#1}
    ,landscape,
    ,fitpaper=true,
%    ,pagecommand={\thispagestyle{lscape}}  
    ,pagecommand={\thispagestyle{empty}}  
  ]{\schematicsFile}%
}




% Machine Code Semantics

\newcommand\mem[1]{\mbox{\bfseries mem}\left[#1\right]}
\newcommand\memmem[1]{\mbox{\bfseries mem}\left[\mbox{\bfseries mem}\left[{#1}\right]\right]}
\newcommand\io[1]{\mbox{\bfseries io}\left[#1\right]}
\newcommand\eq{\leftarrow}

%%%%%%%%%%%%%%%%%%%%%%%%%%%%%%%%%%%%%%%%%%%%%%%%%%%%%%%%%%%%%%%%%%%%%%%%%%%%%%%
%
% DATA STRUCTURES
%
%%%%%%%%%%%%%%%%%%%%%%%%%%%%%%%%%%%%%%%%%%%%%%%%%%%%%%%%%%%%%%%%%%%%%%%%%%%%%%%

% Data structures
\newcommand\ds[1]{{\ttfamily #1\index{#1@{\texttt{#1}}}}}
\makeatletter
\newcommand{\simpledatastructure}[1]{%
  \label{ds:#1}
  \refstepcounter{datastructure}%
  \addcontentsline{lods}{section}{\protect\numberline{\thedatastructure}{\ttfamily #1}}%
  \index{#1@{\texttt{#1}}|(pie}%
  {\textbf{\texttt{#1}:}}
}
\newenvironment{datastructure}[2][Address]{%
  \refstepcounter{datastructure}%
  \addcontentsline{lods}{section}{\protect\numberline{\thedatastructure}{\ttfamily #2}}%
  \index{#2@{\texttt{#2}}|(pie}%
  \label{ds:#1}
  \begin{center}
    \zebrarow{10}
    \begin{longtable}{>{\textbf\bgroup}r<{\egroup}lp{.7\columnwidth}}
      %
      % First header
      %
      \hiderowcolors
      {#1} & Type & Description\\
      \hline
      \noalign{\global\rownum 0\relax}\showrowcolors
      \endfirsthead
      %
      % Subsequent headers
      %
      \hiderowcolors
      \multicolumn{3}{l}{\em Continued from previous page.}\\
      \noalign{\smallskip\smallskip}
      {#1} & Type & Description\\
      \hline
      \noalign{\global\rownum 1\relax}\showrowcolors
      \endhead
      %
      % Footer
      %
      \hiderowcolors
      \hline\noalign{\smallskip\smallskip}
      \multicolumn{3}{r}{\em Continued on next page.}\\
      \endfoot
      %
      % Last footer
      %
      \hiderowcolors
      \hline
      \endlastfoot
      %
      % Content
      %
      \showrowcolors
}{%
    \end{longtable}
  \end{center}%
  \@afterindentfalse%
  \@afterheading%
}
\makeatother
\newcommand\dsdesc[3]{
{#1}&\ds{#2}&{#3}\\
}


%%%%%%%%%%%%%%%%%%%%%%%%%%%%%%%%%%%%%%%%%%%%%%%%%%%%%%%%%%%%%%%%%%%%%%%%%%%%%%%
%
% BITFIELDS
%
%%%%%%%%%%%%%%%%%%%%%%%%%%%%%%%%%%%%%%%%%%%%%%%%%%%%%%%%%%%%%%%%%%%%%%%%%%%%%%%

\newcounter{bitfieldBit}
\makeatletter
\def\bitfieldHeight{0.7}
\def\bitfieldHeightText{0.225}
\def\bitfieldTickMark{0.15}
\def\bitfieldBits{16}

\newenvironment{bitfield@}[1][]{%
  \pgfmathsetmacro{\bitfieldBitsMinusOne}{\bitfieldBits - 1}
  \pgfmathsetmacro{\bitfieldBitsMinusTwo}{\bitfieldBits - 2}
  \pgfmathsetmacro{\bitfieldStep}{\bitfieldWidth / \bitfieldBits}
  \vspace{0.5em}
  \setcounter{bitfieldBit}{0}
  \begin{tikzpicture}
    \draw[fill=white, heavy] (0,0) rectangle (\bitfieldWidth,\bitfieldHeight);
    \foreach \x in {0, ..., \bitfieldBitsMinusOne}{
      \begin{scope}[xshift=\bitfieldWidth cm - \bitfieldStep * (\x cm + 1 cm)]
        \draw(0,0) -- +(0,\bitfieldHeight);
        \draw(\bitfieldStep / 2, \bitfieldHeight / 2) %
        node {\textcond\small\textbf{#1}};
        \draw[color=cftdark!50](\bitfieldStep / 2,\bitfieldHeight) node[above] {\scriptsize\x};
      \end{scope}
      \foreach \x in {0, ..., \bitfieldBitsMinusTwo}{
        \draw[xshift=\bitfieldWidth cm - \bitfieldStep * (\x cm + 1 cm)]%
        (0,\bitfieldHeight) -- +(0, \bitfieldTickMark);
      }
    }
}{%
  \draw[fill=none, heavy] (0,0) rectangle (\bitfieldWidth,\bitfieldHeight);
  \end{tikzpicture}
  \vspace{0.5em}
}
\newenvironment{bitfield}[1][]{%
  \def\bitfieldWidth{10}
  \begin{center}%
    \begin{bitfield@}{#1}%
}{%
    \end{bitfield@}
  \end{center}
}
\newenvironment{cbitfield}[1][]{%
  \def\bitfieldWidth{14}
  \begin{center}%
    \begin{bitfield@}{#1}%
}{%
    \end{bitfield@}
  \end{center}
}

\newenvironment{nbitfield}[2][]{%
  \def\bitfieldWidth{14}
  \def\bitfieldBits{#2}
  \begin{center}%
    \begin{bitfield@}{#1}%
}{%
    \end{bitfield@}
  \end{center}
}


\makeatother

\newcommand\bitfieldItem[3]{%
  \begin{scope}[xshift=\bitfieldWidth cm - \bitfieldStep * (\arabic{bitfieldBit} cm + #1 cm)]
    \draw[fill=#2, draw opacity=0] (0,0) rectangle (\bitfieldStep * #1, \bitfieldHeight);
    \draw(\bitfieldStep * #1 / 2, \bitfieldHeightText) %
    node[anchor=base] {\textcond{\small {#3}}};
    \draw[thick] (0,0) -- +(0, \bitfieldHeight);
    \draw[thick,xshift=\bitfieldStep * #1 cm] (0,0) -- +(0, \bitfieldHeight);
  \end{scope}
  \addtocounter{bitfieldBit}{#1}
}

\newcommand\bitfieldGroup[3]{%
  \begin{scope}[xshift=\bitfieldWidth cm - \bitfieldStep * (\arabic{bitfieldBit} cm + #1 cm)]
    \draw[fill=#2, draw opacity=0] (0,0) rectangle (\bitfieldStep * #1, \bitfieldHeight);
    \draw(\bitfieldStep * #1 / 2, \bitfieldHeightText) %
    node[anchor=base] {\textcond{\small{#3}} };
    \draw[thick] (0,0) -- +(0, \bitfieldHeight);
    \draw[heavy,xshift=\bitfieldStep * #1 cm] (0,0) -- +(0, \bitfieldHeight);
  \end{scope}
  \addtocounter{bitfieldBit}{#1}
}

\newcommand\bitfieldConst[1]{\bitfieldItem{1}{white}{\textbf{#1}}}
\newcommand\bitfieldRepConst[2]{%
  \foreach \x in {1,...,#1} \bitfieldConst{#2};
  \draw[heavy,xshift=\bitfieldStep * #1 cm] (0,0) -- +(0, \bitfieldHeight);
}




% End of file.
