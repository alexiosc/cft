% -*- latex -*-

\definecolor{caution}{RGB}{192,0,0}

\newcommand\textcond{\fontspec{Myriad Pro Condensed}}

% Primary index entries
\newcommand\pie[1]{\textbf{\hyperpage{#1}}}

% Use hyperlinking when rendering PDFs
\newcommand{\barecf}[1]{\hyperref[#1]{\ref*{#1}}}
\newcommand{\cf}[2][section]{\hyperref[#2]{%
        \ifthenelse{\equal{\pageref*{#2}}{\thepage}}%
        {#1 \ref*{#2}}%
        {#1 \ref*{#2} (p.~\pageref*{#2})}%
}}
\newcommand{\cfp}[2][section]{\hyperref[#2]{%
        \ifthenelse{\equal{\pageref*{#2}}{\thepage}}%
        {#1 \ref*{#2}}%
        {#1 \ref*{#2}, p.~\pageref*{#2}}%
}}
\newcommand{\fcf}[1]{\cf[figure]{#1}}
\newcommand{\fcfp}[1]{\cfp[figure]{#1}}
\newcommand{\tcf}[1]{\cf[table]{#1}}
\newcommand{\tcfp}[1]{\cfp[table]{#1}}
\newcommand{\ccf}[1]{\cf[chapter]{#1}}
\newcommand{\ccfp}[1]{\cfp[chapter]{#1}}
%
\newcommand{\npcf}[2][section]{\hyperref[#2]{#1 \ref*{#2}}}
\newcommand{\appcf}[1]{\cf[appendix]{#1}}
\newcommand{\ecf}[1]{\cf[equation]{#1}}
\newcommand{\algcf}[1]{\cf[algorithm]{#1}}
\newcommand{\npappcf}[1]{\npcf[appendix]{#1}}
\newcommand{\npccf}[1]{\npcf[chapter]{#1}}
\newcommand{\npfcf}[1]{\npcf[figure]{#1}}
\newcommand{\nptcf}[1]{\npcf[table]{#1}}
\newcommand{\npecf}[1]{\npcf[equation]{#1}}
\newcommand{\npalgcf}[1]{\npcf[algorithm]{#1}}

\newcommand\hyperemail[1]{\sffamily\href{mailto:#1}{#1}}
\newcommand\link[1]{\sffamily\href{http://#1}{#1}}
\newcommand\ahref[2]{\sffamily\href{#1}{#2}}

\newcommand\op[1]{{\ttfamily #1}}
\newcommand\fwni[1]{{\ttfamily{#1}}}
\newcommand\fw[1]{\fwni{#1}\index{#1@\protect\fwni{#1}}}
%                           \index{#1@\fwni{#1}|(pie}%

\newcommand\f[1]{{\texttt{#1}}}
\newcommand\hex[1]{\textsf{#1}}
\newcommand\bin[1]{\textsf{#1}}
\newcommand\bitmap[1]{\texttt{#1}}
\newcommand\bus[1]{{#1}}
\newcommand\unit[1]{{#1}}
\newcommand\IBUS{\bus{\gls{IBUS}\index{IBUS}}}
\newcommand\DBUS{\bus{\gls{Data Bus}\index{Data Bus}}}
\newcommand\ABUS{\bus{\gls{Address Bus}\index{Address Bus}}}
\newcommand\ALU{\unit{\gls{ALU}\index{ALU}}}
\newcommand\SBU{\unit{\gls{SBU}\index{SBU}}}
\newcommand\AGL{\unit{\gls{AGL}\index{AGL}}}
\newcommand\register[1]{\textsf{#1}\index{Registers!#1}}
\newcommand\A{\register{AC}}
\newcommand\AC{\A}
\newcommand\Areg{\A}
\newcommand\Lreg{\register{L}}
\newcommand\Ireg{\register{I}}
\newcommand\Zreg{\register{Z}}
\newcommand\Vreg{\register{V}}
\newcommand\Nreg{\register{N}}
\newcommand\AR{\register{AR}}
\newcommand\MAR{\AR}
\newcommand\DR{\register{DR}}
\newcommand\PC{\register{PC}}
\newcommand\IR{\register{IR}}

\newcommand{\asm}[1]{\texttt{#1}}
\newcommand{\instr}[1]{\asm{#1}}
\newcommand{\nsni}[1]{$\overline{\mbox{\textsf{{#1}}}}$}
\newcommand{\ns}[1]{\index{#1@{$\protect\overline{\protect\mbox{\textsf{#1}}}$}}%
  \nsni{#1}}
\newcommand{\psni}[1]{\textsf{#1}}
\newcommand{\ps}[1]{\index{#1@\psni{#1}}%
  \psni{#1}}
\newcommand{\lt}[1]{\textsf{#1}}
\newcommand{\sw}[1]{\textsf{#1\index{Switch, front panel!#1}}}
\newcommand{\HC}[1]{\chip{74HC{#1}}}
\newcommand{\HCT}[1]{\chip{74HCT{#1}}}
\newcommand{\chip}[1]{#1\index{#1}}

\newcommand{\schpt}[1]{#1\textsf{#1}}
\newcommand{\farnell}[1]{#1}

\newcommand\BUS[2]{\ps{#1}$_{\mbox{\scriptsize #2}}$}

\newcommand\nBUS[2]{\ns{#1}$_{\mbox{\scriptsize #2}}$}

\newcommand\UINSTR{\ns{uINSTR18}}
\newcommand\HALT{\ns{HALT}}
\newcommand\END{\ns{END}}
\newcommand\IRQ{\ns{IRQ}}
\newcommand\IRQS{\ns{IRQS}}
\newcommand\IRQn[1]{\nBUS{IRQ}{#1}}
\newcommand\RUNITn[1]{\BUS{RUNIT}{#1}}
\newcommand\WUNITn[1]{\BUS{WUNIT}{#1}}
\newcommand\TPA{\ps{TPA}}
\newcommand\TPC{\ps{TPC}}
\newcommand\WAC{\ns{WAC}}
\newcommand\WALU{\ns{WALU}}
\newcommand\WDR{\ns{WDR}}
\newcommand\WIR{\ns{WIR}}
\newcommand\WMAR{\ns{WMAR}}
\newcommand\WPC{\ns{WPC}}
\newcommand\SYSDEV{\ns{SYSDEV}}
\newcommand\IODEV[1]{\ns{IODEV{#1}XX}}
\newcommand\OPIFn[1]{\BUS{OPIF}{#1}}
\newcommand\OPIF{\ps{OPIF}}
\newcommand\GUARDPULSE{\ns{GUARD}}
\newcommand\GP{\GUARDPULSE}
\newcommand\CLOCK[1]{\BUS{CLK}{#1}}
\newcommand\CLKn[1]{\CLOCK{#1}}
\newcommand\RSTHOLD{\ns{RSTHOLD}}
\newcommand\BOE{\ns{BOE}}
\newcommand\UOE{\ns{UOE}}
\newcommand\SKIP{\ns{SKIP}}
\newcommand\AINDEX{\ps{AINDEX}}
\newcommand\CLL{\ns{CLL}}
\newcommand\CPL{\ns{CPL}}
\newcommand\CLI{\ns{CLI}}
\newcommand\STI{\ns{STI}}
\newcommand\IRn[1]{\BUS{IR}{#1}}
\newcommand\PCn[1]{\BUS{PC}{#1}}
\newcommand\IBUSn[1]{\BUS{IBUS}{#1}}
\newcommand\ACn[1]{\BUS{AC}{#1}}
\newcommand\DBUSn[1]{\BUS{DBUS}{#1}}
\newcommand\ABUSn[1]{\BUS{AB}{#1}}
\newcommand\AEXTn[1]{\BUS{AEXT}{#1}}
\newcommand\ISROLL{\ps{ISROLL}}
\newcommand\RAC{\ns{RAC}}
\newcommand\RAGL{\ns{RAGL}}
\newcommand\RDR{\ns{RDR}}
\newcommand\RPC{\ns{RPC}}
\newcommand\INCPC{\ns{INCPC}}
\newcommand\STPAC{\ns{STPAC}}
\newcommand\STPDR{\ns{STPDR}}
\newcommand\INCAC{\STPAC}
\newcommand\INCDR{\STPDR}
\newcommand\DEC{\ns{DEC}}
\newcommand\MEM{\ns{MEM}}
\newcommand\IO{\ns{IO}}
\newcommand\R{\ns{R}}
\newcommand\WRITE{\ns{W}}
\newcommand\WEN{\ns{WEN}}
\newcommand\WAR{\ns{WAR}}
\newcommand\READ{\ns{R}}
\newcommand\FL{\ps{FL}}
\newcommand\FV{\ps{FV}}
\newcommand\FZERO{\ps{FZERO}}
\newcommand\FNEG{\ps{FNEG}}
\newcommand\RESET{\ns{RESET}}
\newcommand\abbr[1]{#1}
\newcommand\SKIPEXT{\ns{SKIPEXT}}
\newcommand\ENDEXT{\ns{ENDEXT}}
\newcommand\WS{\ns{WS}}
\newcommand\UPC{\ps{µPC}}
\newcommand\UCB{\ps{µCB}}
\newcommand\ACCPL{\ns{ACCPL}}

\newcommand\NB{\textbf{Nota Bene:\ }}

\newcommand\bit[1]{{\texttt{#1}}}

\newcommand\notes[1]{{\small\verbatiminput{#1}}}

\newcommand\field[1]{\textsf{#1}}
\newcommand\port[1]{\textsf{#1}}

\newcommand\cftin[1]{\textsf{#1}}
\newcommand\cftout[1]{\textsf{#1}}
\let\cftcode\cftout
\let\cftkbd\cftin

\newcommand\board[1]%
                 {#1\index{Boards!#1}}

\newcommand\li[1]{\item{\bfseries #1}}

% Temporary question environment
\newcommand\question[2]{\textbf{#1} #2}


%%%%%%%%%%%%%%%%%%%%%%%%%%%%%%%%%%%%%%%%%%%%%%%%%%%%%%%%%%%%%%%%%%%%%%%%%%%%%%%
%
% THE I/O PORT AND EXTENDED COMMAND INDEX
%
%%%%%%%%%%%%%%%%%%%%%%%%%%%%%%%%%%%%%%%%%%%%%%%%%%%%%%%%%%%%%%%%%%%%%%%%%%%%%%%

%% \makeatletter
%% \newcommand\ioport@[4]{%
%%   \label{ioport:#1-#4}
%%   \vspace{0.5em}
%%   \noindent\hex{\bfseries{#2}} (\texttt{#1}): {\bfseries\asm{\bfseries{#3}}} — {#4}
%%   \vspace{0.5em}
%% }
%
%
% \ioport{port}{crwvehf}{regname}{descr}
%
%% \newcommand\ioport[4]{%
%%   \ioport@{#1}{#2}{#3}{#4}
%%   \addcontentsline{loioport}{section}{\hex{#2} (\texttt{#1}) \textbf{\asm{#3}} — soup}%
%% }

% \begin{ioport}{VDU}{1F0}{--wvehf}{MCR0}{Mode Control Register 0}
% ...
% \end{ioport}

\newenvironment{ioport}[5]{%
  \vspace{0.5em}
  \addcontentsline{loioport}{section}{\hex{#2} (\texttt{#3}) \textbf{\cftout{#1} \cftout{#4} — #5}}%
  \noindent\hex{\bfseries{#2}} (\texttt{#3}): {\bfseries\asm{\bfseries{#4}}} — {#5}%
  \noindent%
}{%
  \vspace{0.5em}
}

% \begin{extcmd}{PFP}{SR1}{4409}{009}{--wvehf}{Mode Control Register 0}
% ...
% \end{extcmd}
\newenvironment{extcmd}[7]{%
  \vspace{0.5em}
  \addcontentsline{loioport}{section}{\hex{#2} (\texttt{#4}) \textbf{\cftout{#1} \cftout{#2} — #6}}%
  \noindent\hex{\bfseries{#2}} \hex{#3} (I/O port \hex{#4} — \texttt{#5}): {\bfseries{#6}}%
  \noindent%
}{%
  \vspace{0.5em}
}

%\newcommand\extcmda[7]{%
%  \label{ioport:#5-#2}
%  \vspace{0.5em}
%  \noindent\hex{\bfseries{#2}} (\texttt{#1}): {\bfseries\asm{\bfseries{#3}}} — {#4}
%  \vspace{0.5em}
%  \ioport{#4}{#5}{#1}{#7}
%}
%\newcommand\extcmd[7]{% 
%  \extcmda{#1}{#2}{#3}{#4}{#5}{#6}{#7}
%  \addcontentsline{loioport}{section}{\hex{#5} (\texttt{#4}) \textbf{\asm{#1}} — {#6}}%
%  \addcontentsline{loioport}{section}{\protect\numberline{\ }%
%}
\makeatother


%%%%%%%%%%%%%%%%%%%%%%%%%%%%%%%%%%%%%%%%%%%%%%%%%%%%%%%%%%%%%%%%%%%%%%%%%%%%%%%
%
% CODE FOR DISPLAYING AND INDEXING SCHEMATICS
%
%%%%%%%%%%%%%%%%%%%%%%%%%%%%%%%%%%%%%%%%%%%%%%%%%%%%%%%%%%%%%%%%%%%%%%%%%%%%%%%


% \schematic{page number}{description}{label}
\def\schematicsFile{figs/schematics.pdf}
\newcommand\includesch[3]{%
  \stepcounter{subsection}%
  \phantomsection%
  \addcontentsline{toc}{subsection}{\protect\numberline{\thesubsection} #2}%
  \includeschns{#1}{#2}{#3}
}

\newcommand\includeschns[3]{%
  \label{#3}%
  \stepcounter{schematic}%
  \addcontentsline{los}{section}{\protect\numberline{\theschematic} #2}%
  \includepdf[
    pages={#1}
    ,landscape,
    ,fitpaper=true,
%    ,pagecommand={\thispagestyle{lscape}}  
    ,pagecommand={\thispagestyle{empty}}  
  ]{\schematicsFile}%
}


\newcommand\tU{$\uparrow$}
\newcommand\tD{$\downarrow$}

\newcommand\lstkbd[1]{\mathbf{\textbf{#1}}}
%\newcommand\lstfkbd[1]{\underline{\mathbf{\textbf{#1}}}}
\newcommand\lstfkbd[1]{\color{cftoutline}{\mathbf{\textbf{#1}}}}
\lstset{%
        keywordstyle=\fontspec{Inconsolata Bold},%
        keywordstyle=[2]\color{cftoutline}\fontspec{Inconsolata Bold},%
        keywordstyle=[3]\fontspec{Inconsolata Bold},%
        commentstyle=\color{cftlight}%
}
\lstdefinestyle{deb}{mathescape=true,numbers=none}
\lstdefinestyle{forthprogram}{}
\lstdefinelanguage{cftasm}{%
        mathescape=true,
        morekeywords={TRAP,IOT,LOAD,STORE,IN,OUT,JMP,JSR,ADD,AND,OR,%
                      XOR,OP1,OP2,ISZ,LIA,R,I,IFL,IFV,CLA,CLL,NOT,%
                      INC,CPL,RBL,RBR,RNL,RNR,NOP,SNA,SZA,SSL,SSV,SKIP,%
                      SNN,SNZ,SCL,SCV,CLI,SEI,SEL,NEG,ING,LI,SPA,SNP,RET,%
                      RTT,RTI,SBL,SBR},%
        morekeywords=[2]{.equ,.reg,.include,.word,.fill,%
                      .str,.data,.strp,.strn,.page,.macro,.end},%
        alsoletter=.,%
        sensitive=false,%
        morecomment=[l]{/},%
        morecomment=[l]{;},%
}

\lstdefinestyle{longmcasm}{%
        language=mcasm,
        xleftmargin=25pt,
        xrightmargin=5pt,
        framexleftmargin=20pt,
        basicstyle={\footnotesize\ttfamily},
}
\lstdefinelanguage{mcasm}{%
        mathescape=false,
        morekeywords={cond,field,signal,start,hold},%
        morekeywords=[2]{\#define,\#ifdef,\#endif,\#if,\#undef,\#line,\#warning,\#warn,\#error},%
        morekeywords=[3]{INT,RST,V,L,OP,I,SKIP,INC,uaddr},%
        alsoletter=\#,%
        sensitive=false,%
        morecomment=[l]{//},%
        %morecomment=[s]{( }{ )},%
}


\lstdefinelanguage{forth}{%
        mathescape=true,
        %morekeywords={TRAP,IOT,LOAD,STORE,IN,OUT,JMP,JSR,ADD,AND,OR,%
        %              XOR,OP1,OP2,ISZ,LIA,R,I,IFL,IFV,CLA,CLL,NOT,%
        %              INC,CPL,RBL,RBR,RNL,RNR,NOP,SNA,SZA,SSL,SSV,SKIP,%
        %              SNN,SNZ,SCL,SCV,CLI,SEI,SEL,NEG,ING,LI,SPA,SNP,RET,%
        %              RTT,RTI,SBL,SBR},%
        morekeywords=[3]{ok}
        %alsoletter=.,%
        sensitive=false,%
        %morecomment=[l]{\},%
        %morecomment=[s]{( }{ )},%
}

% Machine Code Semantics

\newcommand\mem[1]{\mbox{\bfseries mem}\left[#1\right]}
\newcommand\memmem[1]{\mbox{\bfseries mem}\left[\mbox{\bfseries mem}\left[{#1}\right]\right]}
\newcommand\io[1]{\mbox{\bfseries io}\left[#1\right]}
\newcommand\eq{\leftarrow}

%%%%%%%%%%%%%%%%%%%%%%%%%%%%%%%%%%%%%%%%%%%%%%%%%%%%%%%%%%%%%%%%%%%%%%%%%%%%%%%
%
% DATA STRUCTURES
%
%%%%%%%%%%%%%%%%%%%%%%%%%%%%%%%%%%%%%%%%%%%%%%%%%%%%%%%%%%%%%%%%%%%%%%%%%%%%%%%

% Data structures
\newcommand\ds[1]{{\ttfamily #1\index{#1@{\texttt{#1}}}}}
\makeatletter
\newcommand{\simpledatastructure}[1]{%
  \label{ds:#1}
  \refstepcounter{datastructure}%
  \addcontentsline{lods}{section}{\protect\numberline{\thedatastructure}{\ttfamily #1}}%
  \index{#1@{\texttt{#1}}|(pie}%
  {\textbf{\texttt{#1}:}}
}
\newenvironment{datastructure}[2][Address]{%
  \refstepcounter{datastructure}%
  \addcontentsline{lods}{section}{\protect\numberline{\thedatastructure}{\ttfamily #2}}%
  \index{#2@{\texttt{#2}}|(pie}%
  \label{ds:#1}
  \begin{center}
    \zebrarow{10}
    \begin{longtable}{>{\textbf\bgroup}r<{\egroup}lp{.7\columnwidth}}
      %
      % First header
      %
      \hiderowcolors
      {#1} & Type & Description\\
      \hline
      \noalign{\global\rownum 0\relax}\showrowcolors
      \endfirsthead
      %
      % Subsequent headers
      %
      \hiderowcolors
      \multicolumn{3}{l}{\em Continued from previous page.}\\
      \noalign{\smallskip\smallskip}
      {#1} & Type & Description\\
      \hline
      \noalign{\global\rownum 1\relax}\showrowcolors
      \endhead
      %
      % Footer
      %
      \hiderowcolors
      \hline\noalign{\smallskip\smallskip}
      \multicolumn{3}{r}{\em Continued on next page.}\\
      \endfoot
      %
      % Last footer
      %
      \hiderowcolors
      \hline
      \endlastfoot
      %
      % Content
      %
      \showrowcolors
}{%
    \end{longtable}
  \end{center}%
  \@afterindentfalse%
  \@afterheading%
}
\makeatother
\newcommand\dsdesc[3]{
{#1}&\ds{#2}&{#3}\\
}


%%%%%%%%%%%%%%%%%%%%%%%%%%%%%%%%%%%%%%%%%%%%%%%%%%%%%%%%%%%%%%%%%%%%%%%%%%%%%%%
%
% BITFIELDS
%
%%%%%%%%%%%%%%%%%%%%%%%%%%%%%%%%%%%%%%%%%%%%%%%%%%%%%%%%%%%%%%%%%%%%%%%%%%%%%%%

\newcounter{bitfieldBit}
\makeatletter
\def\bitfieldHeight{0.7}
\def\bitfieldHeightText{0.225}
\def\bitfieldTickMark{0.15}
\def\bitfieldBits{16}

\newenvironment{bitfield@}[1][]{%
  \pgfmathsetmacro{\bitfieldBitsMinusOne}{\bitfieldBits - 1}
  \pgfmathsetmacro{\bitfieldBitsMinusTwo}{\bitfieldBits - 2}
  \pgfmathsetmacro{\bitfieldStep}{\bitfieldWidth / \bitfieldBits}
  \vspace{0.5em}
  \setcounter{bitfieldBit}{0}
  \begin{tikzpicture}
    \draw[fill=white, heavy] (0,0) rectangle (\bitfieldWidth,\bitfieldHeight);
    \foreach \x in {0, ..., \bitfieldBitsMinusOne}{
      \begin{scope}[xshift=\bitfieldWidth cm - \bitfieldStep * (\x cm + 1 cm)]
        \draw(0,0) -- +(0,\bitfieldHeight);
        \draw(\bitfieldStep / 2, \bitfieldHeight / 2) %
        node {\textcond\small\textbf{#1}};
        \draw[color=cftdark!50](\bitfieldStep / 2,\bitfieldHeight) node[above] {\scriptsize\x};
      \end{scope}
      \foreach \x in {0, ..., \bitfieldBitsMinusTwo}{
        \draw[xshift=\bitfieldWidth cm - \bitfieldStep * (\x cm + 1 cm)]%
        (0,\bitfieldHeight) -- +(0, \bitfieldTickMark);
      }
    }
}{%
  \draw[fill=none, heavy] (0,0) rectangle (\bitfieldWidth,\bitfieldHeight);
  \end{tikzpicture}
  \vspace{0.5em}
}
\newenvironment{bitfield}[1][]{%
  \def\bitfieldWidth{10}
  \begin{center}%
    \begin{bitfield@}{#1}%
}{%
    \end{bitfield@}
  \end{center}
}
\newenvironment{cbitfield}[1][]{%
  \def\bitfieldWidth{14}
  \begin{center}%
    \begin{bitfield@}{#1}%
}{%
    \end{bitfield@}
  \end{center}
}

\newenvironment{nbitfield}[2][]{%
  \def\bitfieldWidth{14}
  \def\bitfieldBits{#2}
  \begin{center}%
    \begin{bitfield@}{#1}%
}{%
    \end{bitfield@}
  \end{center}
}


\makeatother

\newcommand\bitfieldItem[3]{%
  \begin{scope}[xshift=\bitfieldWidth cm - \bitfieldStep * (\arabic{bitfieldBit} cm + #1 cm)]
    \draw[fill=#2, draw opacity=0] (0,0) rectangle (\bitfieldStep * #1, \bitfieldHeight);
    \draw(\bitfieldStep * #1 / 2, \bitfieldHeightText) %
    node[anchor=base] {\textcond{\small {#3}}};
    \draw[thick] (0,0) -- +(0, \bitfieldHeight);
    \draw[thick,xshift=\bitfieldStep * #1 cm] (0,0) -- +(0, \bitfieldHeight);
  \end{scope}
  \addtocounter{bitfieldBit}{#1}
}

\newcommand\bitfieldGroup[3]{%
  \begin{scope}[xshift=\bitfieldWidth cm - \bitfieldStep * (\arabic{bitfieldBit} cm + #1 cm)]
    \draw[fill=#2, draw opacity=0] (0,0) rectangle (\bitfieldStep * #1, \bitfieldHeight);
    \draw(\bitfieldStep * #1 / 2, \bitfieldHeightText) %
    node[anchor=base] {\textcond{\small{#3}} };
    \draw[thick] (0,0) -- +(0, \bitfieldHeight);
    \draw[heavy,xshift=\bitfieldStep * #1 cm] (0,0) -- +(0, \bitfieldHeight);
  \end{scope}
  \addtocounter{bitfieldBit}{#1}
}

\newcommand\bitfieldConst[1]{\bitfieldItem{1}{white}{\textbf{#1}}}
\newcommand\bitfieldRepConst[2]{%
  \foreach \x in {1,...,#1} \bitfieldConst{#2};
  \draw[heavy,xshift=\bitfieldStep * #1 cm] (0,0) -- +(0, \bitfieldHeight);
}




% End of file.
