\begin{description}
\item \textbf{Bits 0–3: \UPC}. This is the output of the microcode counter as
  described in~\cf{sec:upc} and represents the offset inside a
  microprogram. The remainder of the bits identify different microprograms.
\item \textbf{Bit 4: \ns{AINDEX}}. Asserted by the Autoindex Logic when autoindex memory locations are accessed. The Autoindex Logic is described in~\cf{sec:ail}.
\item \textbf{Bit 5: \ns{SKIP}}. Asserted by the Skip logic when a microcode
  branch must be performed. One side of the branch may affect flow control at
  the processor level, by appropriate manipulation of the \PC. This is perhaps the most complex and powerful microcode operation. It is discussed in~\cf{sec:sbu}.
\item \textbf{Bit 6: \IRn{11}}. This is R field of instructions, as discussed
  in~\cf{sec:instruction-format}.
\item \textbf{Bits 7–10: \IRn{12–15}}. This is the opcode of the currently
  executing instruction, as discussed in~\cf{sec:instruction-format}.
\item \textbf{Bit 11: \ps{FL}}. The current value of the \Lreg.
\item \textbf{Bit 12: \ps{FV}}. The current value of the overflow flag.
\item \textbf{Bit 13: \ns{IRQS}}. Asserted if interrupts are allowed, an
  interrupt has been received, and a new instruction has just started
  fetching. The signal remains asserted for the duration of the microprogram,
  which is conventionally the microprogram that induces the processor to jump
  to the \gls{Interrupt Service Routine}. This is discussed in detail
  in~\cf{sec:interrupts-state-machine}.
\item \textbf{Bit 14: \ns{RSTHOLD}}. Asserted for a set number of clock cycles after a reset. Used to run the reset microprogram, which initialises certain registers.
\item \textbf{Bits 15–19: \UCB}. This is an optional extension to the microcode
  store. It identifies the microcode bank being used. Each microcode bank
  contains the full microcode of the CFT, with variations allowing for
  different machine revisions to be implemented, or extending the instruction
  set. These bits are normally \bin{0000}. This is discussed in~\cf{sec:ucb}
\end{description}
