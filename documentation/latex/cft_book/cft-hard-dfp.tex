% -*- latex -*-

\HtmlMetaDescription{The Debugging Front Panel Controller and its
  operation.}
%\HtmlMetaGoogleDescription{}
%\HtmlMetaBanner{}
%\HtmlMetaTags{}

\label{chap:hard-dfp}

The Debugging Front Panel is a large sub-project, which replaces both
the Programmer's Front Panel (\ccfp{chap:hard-pfp}) and Debugging
Board (\ccfp{chap:hard-deb}).

The board provides an interface between
the CFT processor, the front panel and an optional computer used to
automate testing the CFT:

\begin{itemize}
\item As with the original PFP, the CFT processor may read from the
  panel switches, internal DIP switches, and output values to the
  Output Register, a group of sixteen lights.

  It may also interact with the attached computer using extended
  instructions intended for testing (such as the assertions
  \asm{SUCCESS} and \asm{FAIL}), and may also output values in various
  forms to the attached machine and input values from the serial
  port. This forms a simple virtual console.

\item The front panel allows a user to operate the computer fully,
  including single-microstep or single-step operation, loading
  programs and debugging and also operating devices in I/O space. 

\item The attached computer is provided with a virtual front panel
  which mirrors most of the physical lights, as well as an interface
  to load programs, run and debug them. This interface may be used
  either automatically by a computer, or manually by a human
  (connected via a terminal). Via the virtual console, a human can
  also use the CFT directly.
\end{itemize}

Like with the DEB board, debugging and console functionality hinges on
a \abbr{MCU}. On the DFP, the MCU is also responsible for debouncing
switches, sequencing memory cycles in a safe manner, and generating
slow clocks used in debugging. This reduces the chip count and wiring
complexity. A number of critical functions are ‘autonomic’ and bypass
the MCU altogether. The run/stop/step/microstep state machine is one
of these subsystems. The panel lights (with two notable exceptions)
are driven directly by the CFT processor and update in real time.

\subsection{Subsystems and Facilities}

The device is made up of more parts and sub-assemblies and cabling
than any other CFT sub-project. There are numerous facilities, both
physical and logical. Originally, these features were planned:

\begin{itemize}

\item The eponymous front panel, the traditional input/output device
  of mini-computers.
  \begin{itemize}
  \item It has 160 lights, though some are left
    unconnected and some are meant for future expansion.
  \item It also has 16 switches used to control, debug and test the computer.
  \item There are also switches to control the front panel itself.
  \item Another 16 switches are directly connected to the \abbr{SR},
    which can be read by both the computer and front panel. They are
    the computer's sole input device.
  \item A key lock switch to control power.
  \item A key lock switch to lock the panel. These locks are mostly
    there for stylistic reasons, since many mini-computers had them.
  \end{itemize}

\item A hidden bank of DIP switches to allow the user to set
  non-volatile preferences. These depend entirely on the CFT's ROM for
  interpretation. These directly control the value of the \abbr{DSR}.

\item Wiring facilities that route the processor boards' front panel
  connectors to the correct front panel lights. These involve a
  large number of connectors and an even larger quantity of wiring.
  
\item More wiring facilities to route signals from add-on cards to
  the lights, including the \abbr{MBU} and \abbr{IRC}.

\item A debouncing facilty for the switches controlling computer
  functions and programming.

\item Mutual lock-out for control switches so that multiple panel
  operations can't be started simultaneously by mistake. This also
  includes locking out most panel operations when the panel lock is
  activated.

\item An optional auto-repeat facility for the Step/Microstep
  switch. This is useful in debugging and saves wear on the switch.

\item A 16-bit Output Register (OR) which displays its value on the
  front panel \abbr{MFD}.

\item A sequencer capable of performing memory and I/O space reads and
  writes, and can optionally increment the address after each
  cycle. This provides a total of eight bus functions, assigned to
  four double-throw toggle switches on the front panel.

\item A switch to configure the optional \abbr{MBU} for either
  bare-metal operation (the computer starts halted, and the entire
  64~kBytes of memory is RAM), or turn-key operation (ROM is mapped to
  the top 32~kBytes, and the computer starts in the run state, like
  all modern computers).

\item A switch to issue panel-initiated interrupt requests for
  interacting with computer programs.

\item A slow clock generator capable of providing two slow clock
  rates, in addition to the rate provided by the CFT processor's own
  clock generator. These are useful for debugging and demonstrations.
  
\item A state machine used to halt the computer at the appropriate
  part of the processor cycle. This state machine is also used for
  stepping and micro-stepping.

\end{itemize}

When the DFP was merged with the DEB board, a number of additional
facilities were added, and many were merged together.

\begin{itemize}

\item foo.

\end{itemize}


\section{The Front Panel}

\section{The Debugging System}

\subsection{Attaching a Computer or Terminal}

The MCU connects to the controlling device via a TTL, non-inverted serial port
terminating in a 6-pin header. The header fits the plug of an FTDI
USB-to-serial cable, but can also be used with an external FTDI USB-to-serial
module. A plain RS-232 connection is possible, but the DFP firmware would have
to be recompiled with bit rate low enough to work with RS-232 connections.

\section{Using the DFP from the CFT}

The DFP appears to the computer as a block of 32 addresses in I/O space. Some
of these addresses are reserved for future expansion. The addresses are meant
to be used as \glspl{extended instruction}.

\section{The DFP Firmware}

% End of file.
