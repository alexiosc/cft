% Forth documentation

\chapter{Introduction}

CFT contains a small implementation of Forth in 16 kWords of ROM,
known as the ROM Forth. This forms the basis of more complex, extended
versions of the language. As is traditional with Forth systems, the
operating system is part of the language, and one extension provides a
shell-like user interface to the operating system without leaving the
Forth interpeter.

In many ways, this is similar to various architectures out there,
including Sun workstations and servers, pre-Intel Apple iMacs, and the
OLPC laptop, all of which contain Open Firmware: an implementation of
Forth that knows enough about the hardware to open a filesystem, read
a file from it, and execute it.

The main difference is that the CFT's ROM Forth {\em is\/} the
operating system, and booting from disk simply loads extensions to
it. In {\em this\/} way, ROM Forth is very much like the ROM BASICs of
home computers in the Eighties. Even with a disk system, the ROM
language remains in operation, it simply acquires various extensions.

\section{Why Forth}

A common question is ‘why Forth? Why not a simple boot loader and a
more modern operating system?’. Here are some answers.

\begin{itemize}
\item Development speed. It's easy\footnote{More or less} to write a
  Forth system, and the basic system provides a powerful interface
  that can be built upon. In comparison, if a boot manager were to be
  written, and then a boot loader, and then an operating system, there
  would be an immense amount of code duplication. While this is common
  on modern PCs, it has little practical use.

\item Simplicity.

\item A tokeniser, parser, shell, compiler, assembler and editor come
  with the core language.

\item The Forth system is {\em very\/} compact. All of Forth 83 can fit in
  approximately 8kW of ROM. Over 400 Forth words and most of the
  operating system can fit in 12 kW.

\item A full programming environment in ROM makes it easier to rescue
  the system, and allows it to be built incrementally: even without a
  disk interface, it's possible to use the language and diagnose
  problems.

\item I like Forth.

\end{itemize}

\section{Terminology Differences: Words and Cells}

Forth introduces a major terminology issue. It uses the term ‘word’ to denote
any Forth identifier: a series of characters that identify a particular piece
of code to run. In other languages, this would be a function, procedure,
subroutine, method, or something similar. In Forth, it's ‘word’.

The CFT, however, is a word-oriented machine. In this sense (and in the sense
employed by computer science at large), ‘word’ denotes the minimum amount of
data the CFT can operate on (a 16-bit value). The Forth term for this is
‘cell’.

To resolve this issue:

\begin{itemize}
\item In chapters discussing Forth, ‘word’ is used in the Forth sense: a token
  or the subroutine identified by that token.
\item ‘Cell’ is used to denote a 16-bit value, the minimum amount of data the
  computer can operate on.
\item ‘Machine word’ is sometimes used to denote the same concept, when some
  hardware-level matters are discussed.
\item ‘Forth word’ may also be used to disambiguate as necessary.
\end{itemize}

\section{Learning Forth}

A Forth tutorial is well beyond the scope of this document. There are numerous
sources of Forth documentation and tutorials online. Perhaps the standard such
text is by {\em Starting FORTH\/} by Leo Brodie, once available in print, but
now available online. It can be found at:

\begin{center}
  \url{http://www.forth.com/starting-forth/}
\end{center}

\section{Forth in a Nutshell}

This represents a tiny introduction to the language.

Forth works interactively. CFT Forth normally issues a \cftout{Ready} prompt
after initialisation or errors, and an \cftout{ok} prompt at the end of every
line it understood, {\em on the same line}.

This last fact represents a cultural shock for many programmers who are used to
the \cftkbd{Enter} key producing an immediate line feed. In Forth, it's the end
of the \cftout{ok} prompt that does this. To help distinguish between user
input and machine output, the part the user types is shown $\lstfkbd{like this}$:

\begin{lstlisting}[mathescape=true,numbers=none]
Ready
$\lstfkbd{1 1 + 1+ .}$  3 ok
$\lstfkbd{FOO}$  FOO  ? Ready
\end{lstlisting}

\noindent Forth interprets user input by splitting it up into tokens separated
by one or more spaces, tabs or newlines. A token may be one of only
three things:

\begin{enumerate}
\item A valid Forth word, which is executed. Forth words may contain
  any non-whitespace characters, including digits, punctuation and
  characters that other languages consider invalid. Addition (\fw{+})
  is a valid Forth word. So is \fw{1+} (increment by one).
\item A number, which is pushed onto the \gls{data stack}.
\item An unknown word, which causes an error.
\end{enumerate}

The average programmer coming from other languages has a ‘is this all?’ moment
reading about Forth because they expect the discussion on grammar and syntax —
it never comes. The only thing to know is: there are numbers, there are words,
and there is white space. A little bit of grammar is introduced as a
side-effect of the way some words work. This will be covered as needed.

\subsection{Stacks}

Forth relies almost exclusively on {\em stacks}. A stack is a last-in first-out
(LIFO) data structure which contains a number of values, and implements (in its
simplest form) a mere two operations: a value can be {\em pushed\/} on top of
the stack, and a value can be {\em popped\/} from the top of the stack.

This is identical to the way a stack of, say, plates works. The first plate to
go on the stack is the last to come off it.

Forth has two stacks: the data stack and the return stack. All numbers
encountered by the Forth interpreter are immediately pushed on the data
stack. Most other words work by popping values off the stack, operating on
them, and possibly pushing the result back onto the stack. The Forth data stack
contains 16-bit values. The interpretation of these values is left to
individual Forth words.

The return stack, which is used internally, keeps track of return addresses, so
words can call each other hierarchically. This allows immense Forth
applications to be built by defining new Forth words in terms of other, simpler
Forth words.

\subsection{Calculations}

Because Forth is stack based, all calculations are in the space and time saving
{\em\gls{postfix}} notation:

\begin{lstlisting}[mathescape=true,numbers=none]
$\lstfkbd{10 20 + 3 * 11 /}$
\end{lstlisting}

\noindent evaluates the expression $((10 + 20) \times 3) / 11$. This is how it
works:

\begin{enumerate}
\item Push 10 onto the stack. The stack now contains: \cftout{10}.
\item Push 20 onto the stack. The stack now contains: \cftout{20 10}.
\item Run the \fw{+} word, which pops two values off the stack, and pushes
  their sum onto it. \fw{+} pushes 30 back onto the stack. The stack now
  contains: \cftout{30}.
\item Push 3 onto the stack. The stack now contains: \cftout{3 30}.
\item Run the \fw{*} word, which pops two values off the top of the stack, and
  pushes their product. At the end of this word, the stack contains:
  \cftout{90}.
\item Push 11 onto the stack. The stack now contains: \cftout{11 90}.
\item Run the \fw{/} word which divides the second topmost item on the stack by
  the topmost one. \fw{/} is an integer division, so it pushes back 8. The
  stack finally contains \cftout{8}, which is the result.
\end{enumerate}

A full tutorial on postfix notation\footnote{also known as Reverse Polish
  Notation, in an allusion to {\em prefix\/} or Polish Notation, invented by
  Polish logician Jan Łukasiewicz in the 1920s.} is beyond the scope of this
chapter. {\em Starting Forth\/} provides a gentle introduction for people who
have never seen it before. An understanding of how postfix works is fundamental
to understanting Forth.

One benefit of postfix is it has almost no syntax. There are no rules of
operator precedence, either. With the exception of some delimiter for numbers
and operators (Forth uses white space), this makes postfix grammar-less. This
property extends to Forth, making it simple to learn.

\subsection{Stack Operations}

Like all stack-based languages, Forth has a great wealth of words that modify
the stack. The absolute essential ones are \fw{DUP}, which duplicates the top
item on the stack; \fw{DROP}, which discards the top item on the stack; and
\fw{SWAP}, which swaps the two topmost items on the stack.

The word \fw{.} (dot) pops a number off the stack and prints it out. Finally,
the word \fw{.s} (dot-stack) prints out the stack left-to-right, starting with
its top-most item.

\begin{lstlisting}[mathescape=true,numbers=none]
$\lstfkbd{1 1 + .}$  2 ok
$\lstfkbd{10 20 +3 * 11 / .}$  8 ok
$\lstfkbd{10 DUP .s}$  10 10 ok
$\lstfkbd{10 20 .s}$  10 20 ok
$\lstfkbd{10 20 SWAP .s}$  20 10 ok
$\lstfkbd{10 20 drop .s}$  10 ok
\end{lstlisting}


\subsection{Errors}

Exceptions are raised (and may be caught by advanced programmers) in a number
of cases. The most obvious ones are when a token is neither a word nor a valid
number:

\begin{lstlisting}[mathescape=true,numbers=none]
$\lstfkbd{foobar}$ FOOBAR  ? Ready
\end{lstlisting}

\noindent Another one seen a lot is \cftout{stack underflow ?}, which occurs when an item
is popped from an empty stack.

Errors clear the stacks, so that a common Forth idiom to clear the stack is to
execute purposefully the non-existent word \fw{X}. This issues an error and
clears the stack. The same can be done using the extant word \fw{QUIT}, which
re-initialises the Forth interpreter.


\subsection{Comments}

Comments may be enclosed in \cftin{(\space}$\dots$\cftin{\space)}. The space
after the open parenthesis is part of the syntax and is not optional:

\begin{lstlisting}[mathescape=true,numbers=none]
$\lstfkbd{( this is a comment )}$
$\lstfkbd{\textbackslash{} this is also a comment}$
$\lstfkbd{(this is not a comment)}$  (THIS  ? Ready.
$\lstfkbd{\textbackslash{}neither is this.}$  \NEITHER  ? Ready.
\end{lstlisting}

\noindent This is because both \fw{(} and \fw{\textbackslash} are simple Forth words that
read input and ignore it up to the closing parenthesis and the end of the line
respectively.

This provides a first glimpse into the ability of Forth to add custom grammar
as necessary. In fact, Forth words may also redefine or extend the parser
algorithm, the interpreter, and even the compiler.

\subsection{Defining words}

New words may be defined by the \fw{:} word using the \cftin{: NEW-WORD ... ;} syntax:

\begin{lstlisting}[mathescape=true,numbers=none]
$\lstfkbd{: HELLO ." Hello, world!" ;}$ ok
\end{lstlisting}

\noindent The word \fw{HELLO} is then added to the dictionary and can subsequently be
executed:

\begin{lstlisting}[mathescape=true,numbers=none]
$\lstfkbd{HELLO}$ Hello, world! ok
\end{lstlisting}

\noindent It is standard practice to include two things as part of a Forth definition:
the \gls{stack effect comment}, and a brief comment that explains what the word does. 

\begin{lstlisting}[mathescape=true,numbers=none]
$\lstfkbd{: HELLO ( -- ) \textbackslash{} Say hello}$  compiled
$  \lstfkbd{." Hello, world!" ;}$  ok
\end{lstlisting}

\noindent The stack effect comment shows what values the word expect to find on the
stack, and what values it leaves on the stack. The two are conventionally
separated by a \texttt{--}. For example, a definition of a \fw{POLYNOMIAL} word to
calculate $x^2 + a$ would be:

\begin{lstlisting}[mathescape=true,numbers=none]
: $\lstfkbd{POLYNOMIAL ( n1 n2 -- n ) \textbackslash{} Calculate n1*n1 + n2}$  compiled
  $\lstfkbd{SWAP DUP * + ;}$  ok
\end{lstlisting}

\noindent The first line of the definition forms a simple means of self-documenting Forth
programs and is similar to docstrings in more recent languages (or Lisp).

The \fw{:} word prepares for a new word definition and switches from the Forth
interpreter to the Forth compiler. The compiler works just like the
interpreter, but instead of pushing numbers onto the stack and executing words,
it adds numbers and words to the word definition. When \cftkbd{Enter} is
pressed while Forth is in compilation mode, \cftout{compiled} is printed. At
the end of the definition, the \fw{;} word completes the definition of the
newly created word, and switches back to the interpreter. As is its habit, the
interpreter acknowledges the end of its input with the familiar \cftout{ok}.

The \cftout{compiled} and \cftout{ok} prompts are useful to include when
showing Forth interaction, but they are not shown when Forth {\em programs\/}
are listed. The small program above would be listed as:

\begin{lstlisting}[style=forthprogram]
: POLYNOMIAL ( n1 n2 -- n ) \ Calculate n1*n1 + n2
  SWAP DUP * + ;
\end{lstlisting}

\section{Differences Between ROM Forth and Forth 83}

\begin{itemize}
  \item The main language vocabulary is named \fw{ROM.FORTH}, not
    \fw{FORTH}. A Forth word \fw{FORTH} is provided that adds all
    necessary vocabularies to the search order. Disk-based Forth
    extensions are expected to redefine this word so that execution
    places \fw{ROM.FORTH} and all additional extension vocabularies on
    the vocabulary stack.
  \item Uses null-terminated bit-packed strings for the dictionary
    instead of counted character-per-cell strings.
  \item There is a limit on the number of defining words that use
    \fw{DOES>}.
\end{itemize}

\chapter{Reference}

This chapter provides a reference of Forth words defined. It is
generated automatically from the Forth source code, and as such, may
contain typesetting errors.\footnote{Which are being corrected.}

\section{List of available Forth Words}

%This is a complete list of defined Forth words. Its value as a
%reference tool is limited unless the online version of the manual is
%used.

\input{../generated/forth-words}

\section{Word Reference}

\newcommand\forthvocabulary[1]{\subsection{#1 Vocabulary}}
\newcommand\forthword[1]{}
\newcommand\forthalias[1]{\def\xforthalias{#1}}
\newcommand\forthflags[1]{}
\newcommand\forthdoc[3]{
  \pagebreak[3]
  \par\noindent{{\ttfamily\textbf{#1}}}%
  \hspace{0.5em} {\ttfamily{#2}}%
  \ifthenelse{\equal{#1}{\xforthalias}}{%
    \index{#1@\protect\fwni{#1}|(pie}%
    %\subsubsection{#1}%
  }{%
    %\subsubsection{#1 (\xforthalias)}%
    \index{\xforthalias@\fwni{\xforthalias}|(pie}%
    \hfill(\xforthalias)%
  }\vspace{0.5ex}%
  \\
  \noindent #3
}
\newcommand\forthcode[1]{

\noindent Defined as:
#1
}
\newcommand\forthenddef{\vspace{2ex}}

\input{../generated/forth-doc}

% End of file.
