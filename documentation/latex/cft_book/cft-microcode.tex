%% This chapter describes the CFT Microcode.

\section{Microcode Theory}

Dieter Müller (Mueller) has a very useful resource on understanding
and writing microcode, and it's suitable for beginners in the
art. It's available at
\url{http://6502.org/users/dieter/mpd/mpd_0.htm}. It forms a very good
introductory document.

Microcode formats are usually split into three major groups:

%% \begin{itemize}
%%   \item{\bfseries Vertical Microcode}, where each micro-instruction is
%%     decoded by the control unit before being able to drive control
%%     signals. Vertical microcode is very compact but requires
%%     additional decoding circuitry (and time).
%%   \item{\bfseries Horizontal Microcode}, where each control signal is
%%     directly encoded in the micro-instruction. Horizontal microcode
%%     formats tend to be extremely wide, but they do not need decoding.
%%   \item{\bfseries Hybrid Formats}, where parts are horizontal and
%%     parts are vertical.
%% \end{itemize}

CFT Microcode is hybrid. A number of fields in the format are
vertically encoded to save bits, but also to avoid bus contention
issues. Consider horizontal microcode to select which unit to enable
for reading from:

%% \begin{center}
%%   \zebra
%%   \begin{tabular}{*{3}{c>{\textsf\bgroup}c<{\egroup}}l}
%%     %\noalign{\smallskip}\hline\noalign{\smallskip}
%%     %\\\hline
%%     A & B & C & Read from\\
%%     %\noalign{\smallskip}\hline\noalign{\smallskip}
%%     \hline
%%     H & H & H & Nothing (all units idle)\\
%%     L & X & X & Unit A\\
%%     X & L & X & Unit B\\
%%     X & X & L & Unit C\\
%%     L & L & L & \textrm{\bfseries Fault: all units (bus contention).}\\
%%     \hline
%%   \end{tabular}
%% \end{center}

The first four cases are valid cases of microcode use. The fourth one,
however, represents one of four cases where multiple units are enabled
for read, causing bus contention and possibly damage to the
processor. This way of storing this information is also highly
wasteful: three bits are used to store four different valid states,
which only uses 50\% of the range afforded by three bits ($2^3 = 8$).

A better case, which completely eliminates the contention issue, is to
encode these bits vertically:

%% \begin{center}
%%   \zebra
%%   \begin{tabular}{*{1}{>{\textsf\bgroup}c<{\egroup}}l}
%%     %\noalign{\smallskip}\hline\noalign{\smallskip}
%%     %\\\hline
%%     SRC & Read from\\
%%     %\noalign{\smallskip}\hline\noalign{\smallskip}
%%     \hline
%%     00 & Nothing (all units idle)\\
%%     01 & Unit A\\
%%     10 & Unit B\\
%%     11 & Unit C\\
%%     \hline
%%   \end{tabular}
%% \end{center}

This format saves one bit, and removes invalid or dangerous
states. This technique is used on the CFT in exactly this way and for
exactly this reason.


\section{The Microcode Assembler}

Most microcode formats are usually assembled using helper software of
various types.  The CFT microcode was prepared using \texttt{mcasm}, a
Microcode Assembler written for this express task, but probably of use
in other similar tasks. The assembler is available online at:

\begin{center}
  \url{http://www.bedroomlan.org/projects/mcasm}
\end{center}

It is written in the Python programming language.\footnote{Itself available at
  \url{http://python.org}, though most modern operating systems bundle it.} The
assembler provides macro and preprocessing facilities using the GNU C
Preprocessor, which is provided with the free GNU C Compiler. Its syntax
borrows numerous elements from C, so it can co-operate with the C
Pre-processor, and so that editor systems can recognise enough of it to provide
syntax highlighting.

The macro facilities allow common tasks like memory cycles to be repeated
without the possibility of minor errors that could introduce processor bugs.

The assembler is capable of providing diagnostic output about the quality of
the code, and generates binary images of as many ROMs as are required to store
the wide micro-instructions.


\begin{figure}
  \centering
  \inputfigure{figure-microcode-store}

  \caption[Microcode structure]{\label{fig:microcode-store2}Microcode
    structure. The Address or Conditional Vector (top) is an input to
    the microcode store. The control vector (bottom) is the Microcode
    store's output. It drives the processor's units. }
\end{figure}


\section{Microcode Address Vector}

The processor's microcode is functionally similar to the cam timer in an
old-style washing machine, or the cylinder of a music box. A program is
executed one step at a time. Each step, one or more units are activated. A
washing machine's cam timer controls the various devices in the washing machine
as a function of time. A music box uses studs to trigger musical instruments as
a function of time.

The microcode could also be visualised as a long table, each column of which
controls one or more units.

The main difference is that the current position in the microcode can jump from
one place to the next. This is a form of flow control.

The CFT microcode has a composite microcode address formed of a number of {\em
  conditionals\/} (in \texttt{mcasm} terminology). This information is also
available in~\cf{sec:microcode-format} and illustrated
in~\fcf{fig:microcode-store2}. In short, the address is a vector of ten fields:

\begin{description}
\item \textbf{Bits 0–3: \UPC}. This is the output of the microcode counter as
  described in~\cf{sec:upc} and represents the offset inside a
  microprogram. The remainder of the bits identify different microprograms.
\item \textbf{Bit 4: \ns{AINDEX}}. Asserted by the Autoindex Logic when autoindex memory locations are accessed. The Autoindex Logic is described in~\cf{sec:ail}.
\item \textbf{Bit 5: \ns{SKIP}}. Asserted by the Skip logic when a microcode
  branch must be performed. One side of the branch may affect flow control at
  the processor level, by appropriate manipulation of the \PC. This is perhaps the most complex and powerful microcode operation. It is discussed in~\cf{sec:sbu}.
\item \textbf{Bit 6: \IRn{11}}. This is R field of instructions, as discussed
  in~\cf{sec:instruction-format}.
\item \textbf{Bits 7–10: \IRn{12–15}}. This is the opcode of the currently
  executing instruction, as discussed in~\cf{sec:instruction-format}.
\item \textbf{Bit 11: \ps{FL}}. The current value of the \Lreg.
\item \textbf{Bit 12: \ps{FV}}. The current value of the overflow flag.
\item \textbf{Bit 13: \ns{IRQS}}. Asserted if interrupts are allowed, an
  interrupt has been received, and a new instruction has just started
  fetching. The signal remains asserted for the duration of the microprogram,
  which is conventionally the microprogram that induces the processor to jump
  to the \gls{Interrupt Service Routine}. This is discussed in detail
  in~\cf{sec:interrupts-state-machine}.
\item \textbf{Bit 14: \ns{RSTHOLD}}. Asserted for a set number of clock cycles after a reset. Used to run the reset microprogram, which initialises certain registers.
\item \textbf{Bits 15–19: \UCB}. This is an optional extension to the microcode
  store. It identifies the microcode bank being used. Each microcode bank
  contains the full microcode of the CFT, with variations allowing for
  different machine revisions to be implemented, or extending the instruction
  set. These bits are normally \bin{0000}. This is discussed in~\cf{sec:ucb}
\end{description}


\section{Microprogram structure}

The microcode store is split into a number of microprograms, each allocated 16
Instructions. The \UPC{} acts as the program counter {\em within\/} a
microprogram, while the remainder of the address vector selects which
microprogram is executed.

Microprograms are prioritised: when the computer is being reset, the value of
the \IR{} is immaterial. If the computer is executing a \asm{LOAD} instruction,
the value of the \ns{SKIP} component of the address vector is immaterial. These
don't-care values are illustrated in~\tcf{table:microprograms}:

%%     \zebrarow{10}
%%     \begin{longtable}{*{8}{>{\textsf\bgroup}c<{\egroup}}l}
%%       %
%%       % First header
%%       %
%%     \hiderowcolors
%%     \ns{RSTHOLD} & \ns{IRQS} & \ps{FV} & \ps{FL} & \IRn{12–15} & \IRn{11} & \ns{SKIP} & \ns{AINDEX} & Microprogram \\
%%     \hline
%%     \noalign{\global\rownum 0\relax}\showrowcolors
%%     \endfirsthead
%%     %
%%     % Subsequent headers
%%     %
%%     \hiderowcolors
%%     \noalign{\smallskip\smallskip}
%%     \multicolumn{9}{l}{\em Continued from previous page.}\\
%%     \ns{RSTHOLD} & \ns{IRQS} & \ps{FV} & \ps{FL} & \IRn{12–15} & \IRn{11} & \ns{SKIP} & \ns{AINDEX} & Microprogram \\
%%     \hline
%%     \noalign{\global\rownum 1\relax}\showrowcolors
%%     \endhead
%%     %
%%     % Footer
%%     %
%%     \hiderowcolors
%%     \hline
%%     \noalign{\smallskip\smallskip}
%%     \multicolumn{9}{l}{\em Continued on next page.}\\
%%     \endfoot
%%     %
%%     % Last footer
%%     %
%%     \hiderowcolors
%%     \hline
%%     \endlastfoot
%%     %
%%     % Content
%%     %
%%     \showrowcolors
%%     L & X & X & X & X & X & X & X & Reset processor\\
%%     H & L & X & X & X & X & X & X & Interrupt handler\\
%%     %
%%     H & H & X & X & \bin{0000} & \bin{0} & X & H & \asm{TRAP} Direct \\
%%     H & H & X & X & \bin{0000} & \bin{1} & X & H & \asm{TRAP} Indirect \\
%%     H & H & X & X & \bin{0000} & \bin{1} & X & L & \asm{TRAP} Autoindex \\
%%     %
%%     H & H & X & X & \bin{0001} & \bin{0} & X & H & \asm{IOT} Direct \\
%%     H & H & X & X & \bin{0001} & \bin{1} & X & H & \asm{IOT} Indirect \\
%%     H & H & X & X & \bin{0001} & \bin{1} & X & L & \asm{IOT} Autoindex \\
%%     %
%%     H & H & X & X & \bin{0010} & \bin{0} & X & H & \asm{LOAD} Direct \\
%%     H & H & X & X & \bin{0010} & \bin{1} & X & H & \asm{LOAD} Indirect \\
%%     H & H & X & X & \bin{0010} & \bin{1} & X & L & \asm{LOAD} Autoindex \\
%%     %
%%     H & H & X & X & \bin{0011} & \bin{0} & X & H & \asm{STORE} Direct \\
%%     H & H & X & X & \bin{0011} & \bin{1} & X & H & \asm{STORE} Indirect \\
%%     H & H & X & X & \bin{0011} & \bin{1} & X & L & \asm{STORE} Autoindex \\
%%     %
%%     H & H & X & X & \bin{0100} & \bin{0} & X & H & \asm{IN} Direct \\
%%     H & H & X & X & \bin{0100} & \bin{1} & X & H & \asm{IN} Indirect \\
%%     H & H & X & X & \bin{0100} & \bin{1} & X & L & \asm{IN} Autoindex \\
%%     %
%%     H & H & X & X & \bin{0101} & \bin{0} & X & H & \asm{OUT} Direct \\
%%     H & H & X & X & \bin{0101} & \bin{1} & X & H & \asm{OUT} Indirect \\
%%     H & H & X & X & \bin{0101} & \bin{1} & X & L & \asm{OUT} Autoindex \\
%%     %
%%     H & H & X & X & \bin{0110} & \bin{0} & X & H & \asm{JMP} Direct \\
%%     H & H & X & X & \bin{0110} & \bin{1} & X & H & \asm{JMP} Indirect \\
%%     H & H & X & X & \bin{0110} & \bin{1} & X & L & \asm{JMP} Autoindex \\
%%     %
%%     H & H & X & X & \bin{0111} & \bin{0} & X & H & \asm{JSR} Direct \\
%%     H & H & X & X & \bin{0111} & \bin{1} & X & H & \asm{JSR} Indirect \\
%%     H & H & X & X & \bin{0111} & \bin{1} & X & L & \asm{JSR} Autoindex \\
%%     %
%%     H & H & X & X & \bin{1000} & \bin{0} & X & H & \asm{ADD} Direct \\
%%     H & H & X & X & \bin{1000} & \bin{1} & X & H & \asm{ADD} Indirect \\
%%     H & H & X & X & \bin{1000} & \bin{1} & X & L & \asm{ADD} Autoindex \\
%%     %
%%     H & H & X & X & \bin{1001} & \bin{0} & X & H & \asm{AND} Direct \\
%%     H & H & X & X & \bin{1001} & \bin{1} & X & H & \asm{AND} Indirect \\
%%     H & H & X & X & \bin{1001} & \bin{1} & X & L & \asm{AND} Autoindex \\
%%     %
%%     H & H & X & X & \bin{1010} & \bin{0} & X & H & \asm{OR} Direct \\
%%     H & H & X & X & \bin{1010} & \bin{1} & X & H & \asm{OR} Indirect \\
%%     H & H & X & X & \bin{1010} & \bin{1} & X & L & \asm{OR} Autoindex \\
%%     %
%%     H & H & X & X & \bin{1011} & \bin{0} & X & H & \asm{XOR} Direct \\
%%     H & H & X & X & \bin{1011} & \bin{1} & X & H & \asm{XOR} Indirect \\
%%     H & H & X & X & \bin{1011} & \bin{1} & X & L & \asm{XOR} Autoindex \\
%%     %
%%     H & H & X & X & \bin{1100} & \bin{0} & H & H & \asm{OP1} Direct, \asm{NOP}s \\
%%     H & H & X & X & \bin{1100} & \bin{1} & L & H & \asm{OP1} Indirect, actions \\
%%     %
%%     H & H & X & X & \bin{1101} & \bin{0} & H & H & \asm{OP2} Direct, \asm{NOP}s \\
%%     H & H & X & X & \bin{1101} & \bin{1} & L & H & \asm{OP2} Indirect, actions \\
%%     %
%%     H & H & X & X & \bin{1110} & \bin{0} & H & H & \asm{ISZ} Direct, no skip \\
%%     H & H & X & X & \bin{1110} & \bin{0} & L & H & \asm{ISZ} Direct, skip \\
%%     H & H & X & X & \bin{1110} & \bin{1} & H & H & \asm{ISZ} Indirect, no skip \\
%%     H & H & X & X & \bin{1110} & \bin{1} & L & H & \asm{ISZ} Indirect, skip \\
%%     H & H & X & X & \bin{1110} & \bin{1} & H & L & \asm{ISZ} Autoindex, no skip \\
%%     H & H & X & X & \bin{1110} & \bin{1} & L & L & \asm{ISZ} Autoindex, skip \\
%%     %
%%     H & H & X & X & \bin{1111} &       X & X & X & \asm{LIA} Immediate \\
%%   \end{longtable}

%%   %% \caption[Table of microprograms]{\label{table:microprograms}Table of
%%   %%   microprograms in the Microprogram Store.}

The don't-care (partial address decoding) values indicate that exactly
half of the microprograms are ‘Reset’ microprograms, one quarter are
‘Interrupt handler’ microprograms, and even many instructions exist in
multiple copies in the microcode store.

\section{Microcode Control Vector}

This information is also available in~\cf{sec:microcode-format}, and
illustrated in~\fcf{fig:microcode-store2}.


\begin{description}
\item \textbf{Bits 0–3: \RUNITn{0–3}}. Vertical field, identifies the processor
  unit to read from, and includes \ALU operations and constants which are ‘read’
  from as if they were separate units operating in tandem.
\item \textbf{Bits 4–6: \WUNITn{0–2}}. Vertical field, identifies the processor unit to write to.
\item \textbf{Bits 7–10: \OPIFn{0–3}}. Vertical field, identifies an
  instruction register bit whose value is routed to the next
  micro-instruction's \ns{SKIP} input.
\item \textbf{Bit 11: \CLL}. Clears \Lreg.
\item \textbf{Bit 12: \CPL}. Complements \Lreg.
\item \textbf{Bit 13: \STI}. Allows interrupts.
\item \textbf{Bit 14: \CLI}. Masks interrupts.
\item \textbf{Bit 15: \INCPC}. Increments \PC.
\item \textbf{Bit 16: \STPDR}. Steps the \DR register.
\item \textbf{Bit 17: \STPAC}. Steps the \AC register.
\item \textbf{Bit 18: \DEC}. When this signal is asserted, \STPDR{}
  and \STPAC{} decrement the \DR{} and \AC{} registers
  respectively. At other times, \STPDR{} and \STPAC{} increment \DR{}
  and \AC{}.
\item \textbf{Bit 19: \MEM}. Indicates a memory cycle.
\item \textbf{Bit 20: \IO}. Indicates an I/O cycle.
\item \textbf{Bit 21: \R}. Indicates a read cycle.
\item \textbf{Bit 22: \WEN}. Indicates a write cycle.
\item \textbf{Bit 23: \END}. Ends the current microprogram.
\end{description}





\section{Microcode State Transitions}

In digital circuits, glitches are short state changes that happen when a
circuit's inputs are moving from one state to another. One benefit of using a
memory chip-based microcode store rather than combinatorial logic is that these
glitches are largely removed.

The potential glitches remaining are logical: as the address vector
changes from one value to another, the values of the outputs must
remain sane.

The state changes of the microcode represent and implement the state
changes of the processor, as described in~\cf{sec:major-states}. Wait
states are not handled by the microcode engine. They cause it to be
suspended temporarily, but this is done by inhibiting \UPC{}
increments and the microcode store is unaware of this. A number of
additional, minor states are accommodated by the microcode: addressing
modes and flow control are the major examples of this.

\subsection{Resetting}

When the \ns{RSTHOLD} conditional is asserted, the microcode address jumps
directly to the {\em middle\/} of the reset microprogram, with the
\UPC{} not reset at all. The reset microprogram is a very simple
affair, however. All it does is load the \PC{} and signal a
microprogram end. This resets the \UPC{}, starting the program over
from the beginning until \ns{RSTHOLD} is cleared.

At that point, the microprogram exits, with the \PC{} appropriately
initialised and the \UPC{} cleared. The next micro-program will be an
instruction fetch for an arbitrary (don't-care) instruction, although
this instruction will always be \bin{0000} (\asm{TRAP}) since the
\IR{} will have been reset to \hex{0000} by the \ns{RESET} signal.

\subsection{Interrupt Handling}

When the \ns{INT} conditional is asserted, the microcode address jumps
directly to the interrupt handler microprogram. Since the \UPC{} is
not affected by this, this would be the middle of the micro-program,
but external registration circuitry ensures that \ns{INT} is only ever
asserted when \UPC{} is zero.

Thus, interrupts are serviced after the end of the previous
instruction and never while carrying it out.

The interrupt handler's first action is to clear interrupts, which
also clears the \ns{INT} conditional. This action is also registered
and is detected by the microcode store at the end of the interrupt
handler microprogram, when \ns{END} is asserted and \UPC{} is reset to
zero. At that point, the interrupt handler microprogram exits and a
new instruction is fetched. (which, since the \PC{} has been set to the
\abbr{ISR} vector, will belong to the macro-program interrupt service
routine)

\subsection{Fetching an Instruction}

The first micro-program cycles of every normal instruction
micro-program contain copies of the exact same fetch microcode. This
code loads the \IR{} from memory, using the current value of the
\PC{}, then increments the \PC{} by one.

The action of loading the \IR{} causes a jump from whatever
instruction's micro-program is executing to a new one. The first
micro-instruction after the end of the fetch cycle begins execution of
the instruction just loaded.

\subsection{Flow Control}

For the most part, the skip logic is idle, which causes \ns{SKIP} to
stay high. Should a micro-instruction request a test of a flag, the
\ns{SKIP} line may be asserted, which will immediately cause a
microprogram jump to a new, non-contiguous address. That address
should perform actions to take skip actions, usually asserting
\ns{INCPC} to increment the \PC{}.

In the next micro-instruction, the \ns{SKIP} conditional will revert
to de-asserted, causing another jump back to the non-skipping version
of the micro-program.

Skipping and non-skipping micro-programs are thus exactly the same
length. Where necessary, this is done by not doing anything, which
causes branching and non-branching instructions to take the same time
to run. This is unusual, since many processors require one less clock
cycle if the branch is not taken. In the case of the CFT, this is less
important since branching is performed by the \asm{OP2} instruction
with performs other optional tasks after the branch (regardless of
whether the branch is taken or not).

It is obvious that flow control happens by jumping from one
micro-program to another without modifying the \UPC{}. This allows
different microcode to be executed. All flow control is performed
using the \OPIF{} microcode field, and affects the microcode
only. Instruction-level flow control is observed as a side-effect of
microcode flow control: simply put, to take an Assembly-level skip,
microcode needs to increment the \PC{}. To not take a skip, the
microcode needs to do nothing.

\subsection{Negative and Overflow skips}

The microcode ROM had a few spare address pins, and these were
connected to the negative and overflow flag registers to allow for
some unforeseen skip instructions. This allowed the \asm{IFV} and
\asm{IFN} sub-instructions to be coded.

These are handled and behave in a way almost identical to that of the
\ns{SKIP} conditional. The unfortunate side effect is that
microprograms that handle both flag conditionals need to be coded four
times, one for each combination of the two flags.

\subsection{Indirection}

Indirect mode is selected by setting \IRn{11}. This is loaded at fetch
time, and is not modified during microprogram execution. Thus, it
works in conjunction with \IRn{12–15} to select among 16 direct and 16
indirect instruction micro-programs (autoindex mode is a variation of
indirect mode).

Instructions \asm{OP1}, \asm{OP2} and \asm{LIA} treat this as a don't
care value, so there are double as many copies of each micro-program
for these instructions.

Indirect mode is updated at the end of the fetch cycle, causing a jump
to the appropriate direct or indirect (or don't care) micro-program.

\subsection{Autoindexing}

Asserting the \ns{AINDEX} conditional causes a jump to the autoindex
versions of instruction microprograms. When \IRn{11} is clear (direct
mode), \ns{AINDEX} is treated as a don't-care value, so is
immaterial. When \IR{11} is set (high), indirect mode is
active. Autoindexing works in indirect mode by executing additional
code at the end of the instruction execution micro-program.

The \ns{AINDEX} conditional is latched based on the value of the
\AR{}. Once set, it remains set until the end of the microprogram,
avoding jumps back to the non-autoindex code. This is necessary since
the \AR{} changes during execution and its new value is likely to
cancel \ns{AINDEX} mid-instruction.

Instructions that observe autoindex mode behave like their indirect
mode versions. At the end of the instruction, additional code
increments the memory location identifyed by the instruction operand.

The microcode sequencer contains a flaw-like limitation to keep
circuitry simple: if an indirect-mode instruction is fetched from an
address in the range \hex{0080}–\hex{00FF} and the previous address
was also an indirect-mode instruction, autoindex mode will be set
regardless of the newly fetched instruction's operand. This is
considered an acceptable limitation, as the autoindex locations are a
limited, highly useful resource, and there is no good reason to be
executing code there.

\subsection{Unusual Glitches}

To avoid unusual glitches, every unused location of the microcode
store asserts \ns{END}, which causes the microprogram to reset. This
should never happen, but if it does, it ensures the processor will
carry on with the next instruction.

\section{Microcode Listing}

\todo{Reinstate the listing.}


%% %{\small
%% \lstinputlisting[style=longmcasm]{../../../microcode/microcode.mc}
%% %}
